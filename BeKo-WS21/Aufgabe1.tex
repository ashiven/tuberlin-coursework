\documentclass[a4paper,onecolumn,oneside,12pt,ngerman]{article}

\usepackage[ngerman]{babel} % language
\usepackage[utf8]{inputenc} % encoding
\usepackage{amsmath,amsfonts,amssymb,amsthm} % mathematical expression, symbols, fonts, and the whole theorem environment engine
\usepackage{xcolor,graphicx} % graphics and colorings
\usepackage{tabularx,booktabs,multirow} % all about tables: booktabs gives the standard formatting, multirow allows cells to span over multiple rows.
\usepackage{scrextend} % super powerful tool that helps a bit with everything
\usepackage[margin=0pt]{subcaption} % specific form for captions
\usepackage[autostyle]{csquotes} % helper for quoting

\usepackage{rotating} % if you want to rotate any object

%=========== graphics, in particular tikz
\usepackage{tikz}
\usetikzlibrary{arrows,decorations.pathmorphing,decorations.pathreplacing,backgrounds,positioning,fit,matrix}
\usetikzlibrary{shapes,calc}


%=========== Pseudocode
\usepackage[linesnumbered,german]{algorithm2e}

%=========== \cref is a powerfull command that automatically states the current environment (e.g. Theorem, Lemma, etc.)
\usepackage[sort&compress,nameinlink,noabbrev,capitalize]{cleveref}


%===========theorem environment
\theoremstyle{plain} % typical theorem-style (bold header, italic font)
\newtheorem{theorem}{Theorem}[section]
\newtheorem{lemma}[theorem]{Lemma}
\newtheorem{corollary}[theorem]{Korollar}
\newtheorem{observation}[theorem]{Beobachtung}
\newtheorem{proposition}[theorem]{Proposition}
\newtheorem{rrule}{Reduktionsregel}[section]

\crefname{rrule}{Reduktionsregel}{Reduktionsregeln} % how cleverref (\cref) has to handle these environments
\Crefname{rrule}{RR}{RRs} % with \Cref us can use what you defined here---somtimes useful if you want to abbreviate environment when citing (e.g. in tables or figures)

\theoremstyle{definition} % typical definition-style (bold header, normal font)
\newtheorem{definition}[theorem]{Definition}

\theoremstyle{remark} % typical theorem-style (italic header, normal font)
\newtheorem{example}{Example}
\newtheorem*{remark}{Remark}
\newtheorem{reduction}{Reduction}
\theoremstyle{plain}

%===========problem definition evironment: gets three arguments (1) problem name #1 (2) input specification #2 (3) question specification #3
\newcommand{\problemdef}[3]{
  \begin{center}
    \begin{minipage}{0.95\textwidth}
      \noindent
      \textsc{#1}
      
      \vspace{2pt}
      \setlength{\tabcolsep}{3pt}
      \begin{tabularx}{\textwidth}{@{}lX@{}}
        \textbf{Eingabe:} 		& #2 \\
        \textbf{Frage:} 	& #3
      \end{tabularx}
    \end{minipage}
  \end{center}
}



%=========== Nützliche Abkürzungen
\newcommand{\NN}{\mathbb{N}} % natürliche Zahlen
\newcommand{\RR}{\mathbb{R}} % reelle Zahlen
\newcommand{\QQ}{\mathbb{Q}} % rationale Zahlen
\newcommand{\ZZ}{\mathbb{Z}} % ganze Zahlen

\title{Hausarbeit im Modul \glqq Berechenbarkeit und Komplexität\grqq\\
\Large{Aufgabe 1}}


%%%%%%%%%%%%%%%%%%%%%%%%%%%%%%%%%%%%%%%%%%%%%%% Ab hier können Sie das Dokument verändern %%%%%%%%%%%%%%%%%%%%%%%%%%%%%%%%%%%%%%%%%%%%%%%%%%%%%%%%%%%%%

% Hier können noch weitere \usepackage{}-Befehle eingebunden werden.

\author{Giulia Benta | Jannik Novak | Tobias Phillip Rimkus} % Hier bitte Ihren Namen eintragen

\begin{document}

\selectlanguage{ngerman}

\maketitle

\section*{Eigenschaften berechenbarer Funktionen}\
\begin{itemize}   
    \item[a)]
    Sei $f(n)$ eine beliebige totale Funktion:
    \begin{align*}
        f : \NN &\to \NN \\
        n &\mapsto f(n)
    \end{align*}
    Es existieren unendlich viele konstante Funktionen:
    \begin{align*}
        g_{i} : \NN &\to \NN \\
        n &\mapsto f(n_{i}) + 1, {n_i} \in \NN, beliebig fest
    \end{align*}

    die größer als $f(n)$ an jeder beliebigen Stelle $n_i$ sind.
    
    $g_{i}(n)$ ist konstant\\$\implies$ $g_{i}(n)$ ist berechenbar\\$\implies$Es existiert keine totale Funktion die größer als alle berechenbaren totalen Funktionen ist.
    
    \item[b)]
Wir betrachten die Busy Beaver Funktion. 
    Für alle Werte $n \geq 4 = n_0$ nimmt die Busy Beaver Funktion so hohe Werte an, dass sie  nicht mehr berechenbar ist. \footnote{Quelle: https://formal.kastel.kit.edu/\~{}beckert/teaching/TheoretischeInformatik-SS07/110707.pdf  Seite 8}
    
    Außerdem ist die BB Funktion streng monoton, d.h. für alle $n \in \NN$ und somit auch für alle $n \geq n_0$ gilt:
    \begin{align*}
        BB(n+1) > BB(n)
    \end{align*}
    Somit ist die BB Funktion ab dem Wert $n_0$ größer als alle berechenbaren totalen Funktionen, da sie streng monoton steigt und ab $n_0$ nicht mehr berechenbar ist.\\$\implies$Es existiert eine totale Funktion die fast größer als alle berechenbaren totalen Funktionen ist.
    \item[c)]
    Sei $f(n)$ eine beliebige totale und berechenbare Funktion:
    \begin{align*}
        f : \NN &\to \NN \\
        n &\mapsto f(n)
    \end{align*}
    und $g(n)$ eine beliebige totale und unberechenbare Funktion:
    \begin{align*}
        g : \NN &\to \NN \\
        n &\mapsto g(n)
    \end{align*}
    
    Wenn $f(n)$ und $g(n)$ fast gleich sind, dann existiert $n_0$, sodass $f(n) = g(n)$ für alle $n > n_0$ .\\$\implies$ Für $n > n_0$ ist $g(n)$ auch berechenbar, weil $g(n) = f(n)$ für alle $n > n_0$ und f(n) ist berechenbar. 
     
     Für $0 \leq n \leq n_0$ nimmt die Funktion $g(n)$ endlich viele Werte an.\\ $g(n)$ heißt unberechenbar falls unendlich viele $n$ existieren, sodass $g(n)$ nicht berechenbar.\\$\implies$ Widerpruch und somit existieren nicht $f$ und $g$, sodass $f$ und $g$ fast gleich sind.
    \item[d)]
    Wir betrachten das spezielle Halteproblem:
    \begin{align*}
	K = \{w\, | \,M_w\, haelt\, auf\, w\}
    \end{align*}
    
    Es existieren unendlich viele Turingmaschinen $M_w$, die halten.\\$\implies$ Die unberechenbare charakteristische Funktion des speziellen Halteproblems, welche zu 1 auswertet wenn ein Wort in K ist und zu 0 auswertet wenn ein Wort nicht in K ist, hat unendlich oft den Funktionswert 1.
    
    Dazu, sei die konstante Funktion $g(n)$:
    \begin{align*}
        g: \NN &\to \NN \\
        n &\mapsto 1
    \end{align*}
    eine totale berechenbare Funktion die immer zu 1 auswertet.\\ $\implies$ die Funktionen sind unendlich oft gleich.\\$\implies$ Es existiert eine totale berechenbare und eine totale unberechenbare Funktion die oft gleich sind.
\end{itemize}
\end{document}
