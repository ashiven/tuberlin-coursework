%%%%%%%%%%%%%%%%%%%%%%%%%%%%%%%%%%%%%%%%%%%%%%%%%%% Hier sollten Sie nichts verändern %%%%%%%%%%%%%%%%%%%%%%%%%%%%%%%%%%%%%%%%%%%%%%%%%%%%%%%%%%%%%%%%
\documentclass[a4paper,onecolumn,oneside,12pt,ngerman]{article}

\usepackage[ngerman]{babel} % language
\usepackage[utf8]{inputenc} % encoding
\usepackage{amsmath,amsfonts,amssymb,amsthm} % mathematical expression, symbols, fonts, and the whole theorem environment engine
\usepackage{xcolor,graphicx} % graphics and colorings
\usepackage{tabularx,booktabs,multirow} % all about tables: booktabs gives the standard formatting, multirow allows cells to span over multiple rows.
\usepackage{scrextend} % super powerful tool that helps a bit with everything
\usepackage[margin=0pt]{subcaption} % specific form for captions
\usepackage[autostyle]{csquotes} % helper for quoting

\usepackage{rotating} % if you want to rotate any object

%=========== graphics, in particular tikz
\usepackage{tikz}
\usetikzlibrary{arrows,decorations.pathmorphing,decorations.pathreplacing,backgrounds,positioning,fit,matrix}
\usetikzlibrary{shapes,calc}


%=========== Pseudocode
\usepackage[linesnumbered,german]{algorithm2e}

%=========== \cref is a powerfull command that automatically states the current environment (e.g. Theorem, Lemma, etc.)
\usepackage[sort&compress,nameinlink,noabbrev,capitalize]{cleveref}


%===========theorem environment
\theoremstyle{plain} % typical theorem-style (bold header, italic font)
\newtheorem{theorem}{Theorem}[section]
\newtheorem{lemma}[theorem]{Lemma}
\newtheorem{corollary}[theorem]{Korollar}
\newtheorem{observation}[theorem]{Beobachtung}
\newtheorem{proposition}[theorem]{Proposition}
\newtheorem{rrule}{Reduktionsregel}[section]

\crefname{rrule}{Reduktionsregel}{Reduktionsregeln} % how cleverref (\cref) has to handle these environments
\Crefname{rrule}{RR}{RRs} % with \Cref us can use what you defined here---somtimes useful if you want to abbreviate environment when citing (e.g. in tables or figures)

\theoremstyle{definition} % typical definition-style (bold header, normal font)
\newtheorem{definition}[theorem]{Definition}

\theoremstyle{remark} % typical theorem-style (italic header, normal font)
\newtheorem{example}{Example}
\newtheorem*{remark}{Remark}
\newtheorem{reduction}{Reduction}
\theoremstyle{plain}

%===========problem definition evironment: gets three arguments (1) problem name #1 (2) input specification #2 (3) question specification #3
\newcommand{\problemdef}[3]{
  \begin{center}
    \begin{minipage}{0.95\textwidth}
      \noindent
      \textsc{#1}
      
      \vspace{2pt}
      \setlength{\tabcolsep}{3pt}
      \begin{tabularx}{\textwidth}{@{}lX@{}}
        \textbf{Eingabe:} 		& #2 \\
        \textbf{Frage:} 	& #3
      \end{tabularx}
    \end{minipage}
  \end{center}
}



%=========== Nützliche Abkürzungen
\newcommand{\NN}{\mathbb{N}} % natürliche Zahlen
\newcommand{\RR}{\mathbb{R}} % reelle Zahlen
\newcommand{\QQ}{\mathbb{Q}} % rationale Zahlen
\newcommand{\ZZ}{\mathbb{Z}} % ganze Zahlen

\title{Hausarbeit im Modul \glqq Berechenbarkeit und Komplexität\grqq\\
\Large{Aufgabe 3}}


%%%%%%%%%%%%%%%%%%%%%%%%%%%%%%%%%%%%%%%%%%%%%%% Ab hier können Sie das Dokument verändern %%%%%%%%%%%%%%%%%%%%%%%%%%%%%%%%%%%%%%%%%%%%%%%%%%%%%%%%%%%%%

% Hier können noch weitere \usepackage{}-Befehle eingebunden werden.

\author{Jannik Novak, Giulia Benta, Tobias Phillip Rimkus} % Hier bitte Ihren Namen eintragen

\begin{document}

\selectlanguage{ngerman}

\maketitle

% Hier beginnt der Hauptteil des Dokuments, in dem Sie Ihre Lösungen zu der Aufgabe formulieren können.

\section*{Turing-Maechtigkeit von MinWHILE}
Beweis durch strukturelle Induktion.\\
\begin{enumerate}
	\item \textbf{Induktionsanfang}\\\\
		Fuer ${i,j,k}\in \{0,1,...,N_Q\}$  $i \neq j$, $i \neq k$, $k \neq j$\\wobei $N_Q + 1$ die Anzahl der benutzten Variablen in einem WHILE- beziehungsweise MinWHILE-Programm Q bezeichnet.\\\\
		Jedes WHILE-Programm der Form:\\ 
\begin{itemize}
	\item[(1)]\begin{algorithm}[H]
 	   $x_i = x_j \pm c$
	\end{algorithm}
\end{itemize}
		kann durch das folgende MinWHILE-Programm simuliert werden:\\
\begin{itemize}
	\item[(2)]\begin{algorithm}[H]
    \While{$x_j \neq 0$}{
        $x_i = x_i + 1$;\\
	$x_k = x_k + 1$;\\	
	$x_j = x_j - 1$
     }
    $x_i = x_i \pm c$;\\
	\While{$x_k \neq 0$}{
        $x_j = x_j + 1$;\\
	$x_k = x_k - 1$
     }
\end{algorithm}
\end{itemize}
		Zu zeigen ist nun, dass das Programm (2) das Programm (1)  simuliert. Programm (1) weist der Variable $x_i$ den Wert der Variable $x_j$  addiert/subtrahiert mit der Konstante $c$ zu. Programm (2) durchlaeuft eine WHILE-Schleife mit Abbruchbedingung 	$x_j = 0$ und inkrementiert die Variablen $x_i$ und $x_k$ mit jedem Schleifendurchlauf um 1/ dekrementiert die Variable $x_j$ mit jedem Schleifendurchlauf um 1. Somit uebertraegt sich der Wert aus $x_j$ in $x_i$ und $x_k$. Anschliessend wird $x_i$ mit der  Konstante c addiert/subtrahiert 	und  $x_i$ zugewiesen. Somit ergibt sich: $x_i = x_j \pm c$ . Zuletzt wird der Variable $x_j$ wieder ihr urspruenglicher Wert zugewiesen indem in einer weiteren WHILE-Schleife der Wert der Variablen $x_k$ auf die Variable $x_j$ uebertragen wird.
	\item \textbf{Induktionsvoraussetzung}\\\\
	Sei P ein WHILE-Programm, so existiert ein MinWHILE-Programm P' welches P simuliert.
	\item \textbf{Induktionsschritt}
	\begin{enumerate}
	\item Seien P1 und P2 zwei WHILE-Programme. Dann ist auch P1;P2 ein WHILE-Programm. Wenn P1' und P2' MinWHILE-Programme sind die P1 und P2 simulieren, so simuliert das MinWHILE-Programm P1';P2' das WHILE-Programm P1;P2 .
	\item Wenn P ein MinWHILE-Programm ist, dann ist auch:\\
\begin{algorithm}[H]
    \While{$x_i \neq 0$}{
	P
     }
\end{algorithm}
	ein MinWHILE-Programm, welches P ausfuehrt und nach Ausfuehrung von P die Schleifenbedingung $x_i \neq 0$ prueft.
	\item Wenn P ein WHILE-Programm ist, dann ist auch:\\
	\begin{align*}
		LOOP\quad x_i\quad DO\quad P\quad END
	\end{align*} 
	ein WHILE-Programm, welches P $x_i$ mal ausfuehrt.
	\item Sei P ein WHILE-Programm und P' ein MinWHILE-Programm welches P simuliert. Betrachten wir nun das WHILE-Programm aus (c):
	\begin{itemize}
	\item[(1)] \begin{align*}
			LOOP\quad x_i\quad DO\quad P\quad END
		\end{align*} 
	\end{itemize}
	Dann koennen wir dieses WHILE-Programm mithilfe des MinWHILE-Konstrukts aus (b) wie folgt simulieren:\\
	\begin{itemize}
	\item[(2)]	\begin{algorithm}[H]
    			\While{$x_i \neq 0$}{
				P';\\
				$x_i = x_i - 1$
    			 }
		\end{algorithm}
	\end{itemize}
	Zu zeigen ist nun, dass das Programm (2) das Programm (1) simuliert. Programm (1) fuehrt das Programm P $x_i$ mal aus. Programm (2) prueft vor jeder Ausfuehrung ob $x_i \neq 0$ und bricht ab sobald $x_i = 0$. In jedem Schleifendurchlauf wird die Variable $x_i$ um 1 dekrementiert, solange bis die Abbruchbedingung $x_i = 0$ erreicht ist. Somit wird das Programm P' $x_i$ mal ausgefuehrt. Da P' das Programm P simuliert, gilt die obige Aussage.
	\end{enumerate}
\end{enumerate}



\end{document}
