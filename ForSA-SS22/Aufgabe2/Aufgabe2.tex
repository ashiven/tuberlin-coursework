\documentclass[twoside,10pt,fleqn,headinclude=false]{scrartcl}

\usepackage[utf8]{inputenc}
\usepackage[T1]{fontenc}
\usepackage[ngerman]{babel}
\usepackage{csquotes}
\usepackage{xifthen}
\usepackage{textcomp, xspace, url, amsmath, booktabs, amsfonts, amssymb, version, ntheorem, lastpage, scrtime, pifont, wasysym, tabularx, semantic}
\usepackage{pgf}
\usepackage{tikz}
	\usetikzlibrary{trees, decorations, arrows, automata, shadows, positioning, plotmarks, calc, matrix}
\usepackage[arrow, matrix, curve]{xy}
%\usepackage{scrpage2}
\usepackage[xspace]{ellipsis}
\usepackage{palatino}
\usepackage{ragged2e}
\usepackage[tracking=true]{microtype}
\usepackage{multirow}
\usepackage{paralist}
\usepackage{parskip}
\usepackage[footskip=1cm,bindingoffset=2cm,hmargin={0cm, 0cm}, inner=1cm, outer=3cm, top=2cm, bottom=2cm]{geometry}
	\setlength{\marginparwidth}{2cm}
	\renewcommand{\arraystretch}{1.2}
	\setlength{\arrayrulewidth}{1pt}

\usepackage{ForSA.Macros}

%%%%%%%%%%%%%%%
%  meta data  %
%%%%%%%%%%%%%%%

\author{Edgar Arndt}
\def\xdate{SoSe 2022}

%%%%%%%%%%%%%%%%%%%%%%%%%%
%  counter declarations  %
%%%%%%%%%%%%%%%%%%%%%%%%%%

\newcounter{firstTask}
\newcounter{lastTask}
\newcount\member


% Nummer der ersten Aufgabe in dieser HA-Abgabe
\setcounter{firstTask}{01}
% Nummer der letzten Aufgabe in dieser HA-Abgabe
\setcounter{lastTask}{04}
% durch Kommata getrennte Liste aller Punktzahlen mit einem Komma am Ende
\def\xpoints{19,14,17,10,}

\begin{document}
\pagestyle{empty}

%%%%%%%%%%%%
%  header  %
%%%%%%%%%%%%

\centering\normalfont\Large{\textsc{Formale Sprachen und Automaten}}\\
\raggedright\normalsize{\textsl{MTV: Modelle und Theorie Verteilter Systeme}\hfill\textsl{\xdate}}\\
\centering\Large{Hausaufgabe}\\
\vspace{0.5\baselineskip}\centering\large{Name: Jannik Leander Hyun Ho Novak}\\
\vspace{0.5\baselineskip}\centering\large{Matrikelnummer: 392210}\\
\vspace{0.5\baselineskip}\centering\large{(optional) Name: Pete Schimkat}\\
\vspace{0.5\baselineskip}\centering\large{(optional) Matrikelnummer: 403246}


\newlength{\MaxLength}
	\settowidth{\MaxLength}{\textsc{Korrektur}}
\newcount\taskscount
	\taskscount=\value{lastTask}
	\advance\taskscount by -\value{firstTask}
	\advance\taskscount by 1
\newcount\columncount
	\columncount=\taskscount
	\advance\columncount by 2
\newlength{\itemwidth}
	\settowidth{\itemwidth}{1000}

\def\tabletasks{}{
	\member=\value{firstTask}
	\advance\member by -1
	\loop
		\ifnum\member < \value{lastTask}
		\advance\member by 1
		\appto\tabletasks{& \centering}
		\xappto\tabletasks{\the\member}
	\repeat
}

\newcounter{pointsum}
\setcounter{pointsum}{0}
\def\tablepoints#1,#2\END{
	& \centering \ #1
	\addtocounter{pointsum}{#1}
	\ifx\relax#2\relax\else
	   \tablepoints #2\END
	\fi
}

\def\tablecells{}{
	\member=\value{firstTask}
	\advance\member by -1
	\loop
		\ifnum\member < \value{lastTask}
		\advance\member by 1
		\xappto\tablecells{&}
	\repeat
}


$ $
\vspace{\fill}

\flushleft
Je 3 erreichte Hausaufgabenpunkte entsprechen einem Portfoliopunkt. Es wird mathematisch auf halbe Portfoliopunkte gerundet. 
\vspace{2em}\\
\textbf{Korrektur:} \vspace{0.5em}\\
\begin{tabular}{|>{\scshape}p{\MaxLength}|*{\taskscount}{p{\itemwidth}|}|p{\itemwidth}|}
	\hline
	\textsc{Aufgabe} \tabletasks & {\centering \ $ \sum $}\\
	\hline
	\textsc{Punkte}
	\expandafter\tablepoints \xpoints\END & {\centering \ \arabic{pointsum}}\\
	\hline
	\textsc{Erreicht} \tablecells &\\
	\hline
	\textsc{Korrektur} \tablecells &\\
	\hline
	\multicolumn{\columncount}{|l|}{Portfoliopunkte:}\\
	\hline
\end{tabular}
\vspace{\fill}
\newpage

%%%%%%%
% Eidesstattliche Erklärung
%%%%%%%
\section*{Erklärung über Arbeitsteilung}

Hiermit versichern wir, dass wir alle Aufgaben zu je gleichen Teilen bearbeitet und die vorliegenden Lösungen zu je gleichen Teilen erstellt haben.

\begin{align*}
	& \underline{\hspace{5cm}} && \underline{\hspace{5cm}}\\
	& \text{ Novak, Jannik Leander Hyun Ho} && \text{Ort, Datum}
\end{align*}

\begin{align*}
	& \underline{\hspace{5cm}} && \underline{\hspace{5cm}}\\
	& \text{Schimkat, Pete} && \text{Ort, Datum}
\end{align*}

\newpage
%%%%%%%
% Lösungen zur Aufgabe 4
%%%%%%%

\textbf{Aufgabe 4: Pumping Lemma regulärer Sprachen} \hfill \textbf{19 Punkte}
\begin{compactenum}
	\item[4a)] Sei $n \in \N$ (beliebig aber fest). Wir waehlen das Wort $w = 1^{n}001^{n}$, denn mit $x=1^{n-1}$ ist $w = 1^{n}001^{n}$ und $w \in A_1$. Es gilt weiterhin $\vert w\vert = 2n + 2 \geq n$. Sei $w=xyz$ eine beliebige Zerlegung mit $y\neq \varepsilon$ und $\vert xy\vert \leq n$. Dann ist $x=1^i$, $y=1^j$ und $z=1^{n-i-j}001^n$ fuer ein $j \neq 0$ und $i+j\leq n$. Wir waehlen $k = 0$. Dann ist $xy^{0}z = 1^{i} \varepsilon1^{n-i-j}001^n=1^{n-j}001^n \not \in A_1$, da $j \neq 0$. Da $\neg$\textbf{PUMPING-REG}$(A_1)$, ist $A_1$ nach dem Pumping-Lemma nicht regulaer.
	\item[] \hfill \textbf{9 Punkte}
	\item[4b)] Sei $n \in \N$ (beliebig aber fest). Wir waehlen das Wort $w = 0^{n}11(01)^{2n+1}$, denn $1 \in \N^{+}$ und $2n+1 > 2n$ und somit $w \in A_2$. Es gilt weiterhin $\vert w\vert = n +2 + 2(2n+1)=5n+4 \geq n$. Sei $w=xyz$ eine beliebige Zerlegung mit $y\neq \varepsilon$ und $\vert xy\vert \leq n$. Dann ist $x=0^i$, $y=0^j$ und $z=0^{n-i-j}11(01)^{2n+1}$ fuer ein $j \neq 0$ und $i+j\leq n$. Wir waehlen $k = 2$. Dann ist $xy^{2}z = 0^{i}0^{2j}0^{n-i-j}11(01)^{2n+1}=0^{n+j}11(01)^{2n+1} \not \in A_2$, denn  $2(n+j) = 2n+2j > 2n+1$ fuer $j \neq 0$. Da $\neg$\textbf{PUMPING-REG}$(A_2)$, ist $A_2$ nach dem Pumping-Lemma nicht regulaer.
	\item[] \hfill \textbf{10 Punkte}

\end{compactenum}

\newpage
%%%%%%%%
% Lösungen zur Aufgabe 5
%%%%%%%%

\textbf{Aufgabe 5: Myhill-Nerode für reguläre Sprachen} \hfill \textbf{14 Punkte}
\begin{compactenum}
	\item[5a)] $[a]_{\equiv A} = \{ (bb)^na(bb)^m \vert n,m \in \N \}$ \\$[b]_{\equiv A} = \{ b(bb)^n \vert n \in \N \}$ \\$[ab]_{\equiv A} =\{(bb)^nab(bb)^m \vert n,m \in \N \}$ \\$[bb]_{\equiv A} =\{(bb)^n \vert n \in \N \}$ \\$[ba]_{\equiv A} =\{b(bb)^nay \vert y \in \Sigma^{*} \wedge n \in \N \} \cup \{w \vert w \in \Sigma^{*} \wedge \vert w \vert_a > 1\}$
	\item[] \hfill \textbf{7 Punkte}
	\item[5b)] $M_A=(\{[a],[b],[ab],[bb],[ba]\},\Sigma,\delta_A,[bb],\{[ab],[bb]\})$, \\ $\delta_A=\{([a],a,[ba]),([a],b,[ab]),([b],a,[ba]),([b],b,[bb]),([ab],a,[ba]),([ab],b,[a]),$\\$([bb],a,[a]),([bb],b,[b]),([ba],a,[ba]),([ba],b,[ba])\}$
	\item[] \hfill \textbf{7 Punkte}

\end{compactenum}

\newpage

%%%%%%%%
% Lösungen zur Aufgabe 6
%%%%%%%%

\textbf{Aufgabe 6: Myhill-Nerode für nicht-reguläre Sprachen} \hfill \textbf{17 Punkte}
\begin{compactenum}
	\item[] $[a^m]_{\equiv B} = \{a^m\}$ fuer $m \in \N^{+}$ \\
	$[a^{l}b]_{\equiv B} = \{ a^la^nbwb^n \vert n \in \N \wedge w \in \{a,ab\}^{*} \wedge \neg P_{b}(w)\}$ fuer $l \in \N^{+}$\\
	$[b]_{\equiv B} = \{b\} \cup \{a^nbwb^n \vert n \in \N^{+} \wedge w \in \{a,ab\}^{*} \wedge P_{b}(w)\}$ \\
	$[bb]_{\equiv B} = B \cup \{bw \vert w \in \Sigma^{+}\}$ \\ 
	$[\varepsilon]_{\equiv B}=\{\varepsilon\}$ \\
	
\end{compactenum}

\newpage
%%%%%%%%
% Lösungen zur Aufgabe 7
%%%%%%%%

\textbf{Aufgabe 7: Beschreiben und erzeugen von regulären Sprachen} \hfill \textbf{10 Punkte}
\begin{compactenum}
	\item[7a)] (A1) A ist nicht regulär. \\
	(Z1) B ist nicht regulär. \\
	Beweis über Widerspruchsannahme: \\
	(A2) B ist regulär. \\
	Wir vereinigen die Mengen A und B zu $AB=A\cup B \triangleq \{a^{2n}b^{m}c^{k}\vert m,n,k \in \N \land (n+m \leq k \lor n+m > k)\}$ \\
	$\JustImp{Tautologie} \ \ \ \ \ \ AB \triangleq \{a^{2n}b^{m}c^{k}\vert m,n,k \in \N\}$ \\
	Wir zeigen, dass AB regulär ist, indem wir einen endlichen Automaten konstruieren, der die Sprache akzeptiert: \\
	\begin{center}
    \begin{tikzpicture}[scale=0.2]
    \tikzstyle{every node}+=[inner sep=0pt]
    \draw [black] (19,-28.1) circle (3);
    \draw (19,-28.1) node {$q0$};
    \draw [black] (19,-28.1) circle (2.4);
    \draw [black] (57.1,-28.1) circle (3);
    \draw (57.1,-28.1) node {$q3$};
    \draw [black] (57.1,-28.1) circle (2.4);
    \draw [black] (38.6,-28.1) circle (3);
    \draw (38.6,-28.1) node {$q2$};
    \draw [black] (38.6,-28.1) circle (2.4);
    \draw [black] (19,-14.5) circle (3);
    \draw (19,-14.5) node {$q1$};
    \draw [black] (22,-28.1) -- (35.6,-28.1);
    \fill [black] (35.6,-28.1) -- (34.8,-27.6) -- (34.8,-28.6);
    \draw (28.8,-27.6) node [above] {$b$};
    \draw [black] (41.6,-28.1) -- (54.1,-28.1);
    \fill [black] (54.1,-28.1) -- (53.3,-27.6) -- (53.3,-28.6);
    \draw (47.85,-27.6) node [above] {$c$};
    \draw [black] (37.277,-25.42) arc (234:-54:2.25);
    \draw (38.6,-20.85) node [above] {$b$};
    \fill [black] (39.92,-25.42) -- (40.8,-25.07) -- (39.99,-24.48);
    \draw [black] (55.777,-25.42) arc (234:-54:2.25);
    \draw (57.1,-20.85) node [above] {$c$};
    \fill [black] (58.42,-25.42) -- (59.3,-25.07) -- (58.49,-24.48);
    \draw [black] (54.867,-30.101) arc (-51.33989:-128.66011:26.919);
    \fill [black] (54.87,-30.1) -- (53.93,-30.21) -- (54.55,-30.99);
    \draw (38.05,-36.5) node [below] {$c$};
    \draw [black] (17.563,-25.476) arc (-158.74175:-201.25825:11.519);
    \fill [black] (17.56,-17.12) -- (16.81,-17.69) -- (17.74,-18.05);
    \draw (16.28,-21.3) node [left] {$a$};
    \draw [black] (20.699,-16.959) arc (25.97233:-25.97233:9.913);
    \fill [black] (20.7,-25.64) -- (21.5,-25.14) -- (20.6,-24.7);
    \draw (22.2,-21.3) node [right] {$a$};
    \draw [black] (7.7,-28.1) -- (16,-28.1);
    \fill [black] (16,-28.1) -- (15.2,-27.6) -- (15.2,-28.6);
    \end{tikzpicture}
    \end{center}
    Es gilt, dass AB und B (Annahme A2) regulär sind. Nach Theorem 2.4.4. gilt außerdem, dass auch $AB\setminus B$ regulär ist. Da $AB = A \cup B$ gilt, ist per Definition  $AB\setminus B = A$. Somit würde daraus folgen, dass auch A regulär ist. \\
    Dies widerspricht unserer Annahme A1, Somit muss unsere Widerspruchsannahme falsch sein. \\
    Es gilt, dass B nicht regulär ist. 
	\item[] \hfill \textbf{4 Punkte}
	\item[7b)] Es gelten das Alphabet $\Sigma = \{a,b,c\}$ und die Grammatik $G_B = (\{S,T\},\Sigma,P,S$) mit: \\
	$G_B : S -> aA_1 | bBc | bc | C | \epsilon$ \\
	$A_1 -> aA_2c |ac $\\
	$A_2 -> aA_1 | bBc | bc | C$ \\
	$B -> bBc | bc | C$ \\
	$C -> cC | c$ \\
	Die Grammatik ist vom Typ 2. 
	\item[] \hfill \textbf{6 Punkte}

\end{compactenum}



\end{document}


