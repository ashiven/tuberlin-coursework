\documentclass[twoside,10pt,fleqn,headinclude=false]{scrartcl}

\usepackage[utf8]{inputenc}
\usepackage[T1]{fontenc}
\usepackage[ngerman]{babel}
\usepackage{csquotes}
\usepackage{xifthen}
\usepackage{textcomp, xspace, url, amsmath, booktabs, amsfonts, amssymb, version, ntheorem, lastpage, scrtime, pifont, wasysym, tabularx, semantic}
\usepackage{pgf}
\usepackage{tikz}
	\usetikzlibrary{trees, decorations, arrows, automata, shadows, positioning, plotmarks, calc, matrix}
\usepackage[arrow, matrix, curve]{xy}
%\usepackage{scrpage2}
\usepackage[xspace]{ellipsis}
\usepackage{palatino}
\usepackage{ragged2e}
\usepackage[tracking=true]{microtype}
\usepackage{multirow}
\usepackage{paralist}
\usepackage{parskip}
\usepackage[footskip=1cm,bindingoffset=2cm,hmargin={0cm, 0cm}, inner=1cm, outer=3cm, top=2cm, bottom=2cm]{geometry}
	\setlength{\marginparwidth}{2cm}
	\renewcommand{\arraystretch}{1.2}
	\setlength{\arrayrulewidth}{1pt}

\usepackage{ForSA.Macros}

%%%%%%%%%%%%%%%
%  meta data  %
%%%%%%%%%%%%%%%

\author{Edgar Arndt}
\def\xdate{SoSe 2022}

%%%%%%%%%%%%%%%%%%%%%%%%%%
%  counter declarations  %
%%%%%%%%%%%%%%%%%%%%%%%%%%

\newcounter{firstTask}
\newcounter{lastTask}
\newcount\member


% Nummer der ersten Aufgabe in dieser HA-Abgabe
\setcounter{firstTask}{01}
% Nummer der letzten Aufgabe in dieser HA-Abgabe
\setcounter{lastTask}{03}
% durch Kommata getrennte Liste aller Punktzahlen mit einem Komma am Ende
\def\xpoints{16,16,8,}

\begin{document}
\pagestyle{empty}

%%%%%%%%%%%%
%  header  %
%%%%%%%%%%%%

\centering\normalfont\Large{\textsc{Formale Sprachen und Automaten}}\\
\raggedright\normalsize{\textsl{MTV: Modelle und Theorie Verteilter Systeme}\hfill\textsl{\xdate}}\\
\centering\Large{Hausaufgabe}\\
\vspace{0.5\baselineskip}\centering\large{Name: Jannik Leander Hyun-Ho Novak}\\
\vspace{0.5\baselineskip}\centering\large{Matrikelnummer: 392210}\\
\vspace{0.5\baselineskip}\centering\large{(optional) Name: Pete Schimkat}\\
\vspace{0.5\baselineskip}\centering\large{(optional) Matrikelnummer: 403246}


\newlength{\MaxLength}
	\settowidth{\MaxLength}{\textsc{Korrektur}}
\newcount\taskscount
	\taskscount=\value{lastTask}
	\advance\taskscount by -\value{firstTask}
	\advance\taskscount by 1
\newcount\columncount
	\columncount=\taskscount
	\advance\columncount by 2
\newlength{\itemwidth}
	\settowidth{\itemwidth}{1000}

\def\tabletasks{}{
	\member=\value{firstTask}
	\advance\member by -1
	\loop
		\ifnum\member < \value{lastTask}
		\advance\member by 1
		\appto\tabletasks{& \centering}
		\xappto\tabletasks{\the\member}
	\repeat
}

\newcounter{pointsum}
\setcounter{pointsum}{0}
\def\tablepoints#1,#2\END{
	& \centering \ #1
	\addtocounter{pointsum}{#1}
	\ifx\relax#2\relax\else
	   \tablepoints #2\END
	\fi
}

\def\tablecells{}{
	\member=\value{firstTask}
	\advance\member by -1
	\loop
		\ifnum\member < \value{lastTask}
		\advance\member by 1
		\xappto\tablecells{&}
	\repeat
}


$ $
\vspace{\fill}

\flushleft
Je 4 erreichte Hausaufgabenpunkte entsprechen einem Portfoliopunkt.\vspace{2em}\\
\textbf{Korrektur:} \vspace{0.5em}\\
\begin{tabular}{|>{\scshape}p{\MaxLength}|*{\taskscount}{p{\itemwidth}|}|p{\itemwidth}|}
	\hline
	\textsc{Aufgabe} \tabletasks & {\centering \ $ \sum $}\\
	\hline
	\textsc{Punkte}
	\expandafter\tablepoints \xpoints\END & {\centering \ \arabic{pointsum}}\\
	\hline
	\textsc{Erreicht} \tablecells &\\
	\hline
	\textsc{Korrektur} \tablecells &\\
	\hline
	\multicolumn{\columncount}{|l|}{Portfoliopunkte:}\\
	\hline
\end{tabular}
\vspace{\fill}
\newpage

%%%%%%%
% Eidesstattliche Erklärung
%%%%%%%
\section*{Erklärung über Arbeitsteilung}

Hiermit versichern wir, dass wir alle Aufgaben zu je gleichen Teilen bearbeitet und die vorliegenden Lösungen zu je gleichen Teilen erstellt haben.

\begin{align*}
	& \underline{\hspace{5cm}} && \underline{\hspace{5cm}}\\
	& \text{Novak, Jannik Leander Hyun-Ho} && \text{Ort, Datum}
\end{align*}

\begin{align*}
	& \underline{\hspace{5cm}} && \underline{\hspace{5cm}}\\
	& \text{Schimkat, Pete} && \text{Ort, Datum}
\end{align*}

\newpage
%%%%%%%
% Lösungen zur Aufgabe 1
%%%%%%%

\textbf{Aufgabe 1: Beweismethoden} \hfill \textbf{16 Punkte}
\begin{compactenum}
	\item[1a)] Wir bezeichnen im Folgenden das $i$-te Element der Menge $M$ mit $m_i$, $i \in \N$.\\
	\textbf{Induktionsanfang}\\
	$m_0$ mod $2 = 5 \cdot (2 \cdot 0 + 1 )$ mod $2 = 5$ mod $2 = 1$\\
	\textbf{Induktionsvorraussetzung}\\
	Fuer ein beliebiges aber festes $n \in \N$ gelte: $m_n$ mod $2 = 1$.\\
	\textbf{Induktionsschritt}\\
	$m_{n+1}$ mod $2$\\
	$= 5 \cdot(2(n+1) + 1)$ mod $2$\\
	$= (10n + 15)$ mod $2$\\
	$= ((10n $ mod $2) + (15 $ mod $2))$ mod $2$\\
	$=(0 + 1 )$ mod $2$\\
	$= 1$ mod $2$\\
	$= 1$\\
	Somit gilt $\forall m \in M. m$ mod $2=1$.
	\item[] \hfill \textbf{7.5 Punkte}
	\item[1b)] $\lnot( (\exists x P_1(x) \rightarrow \lnot P_2(x) ) \rightarrow ( \exists y \lnot ( P_2(y) \land P_1(y) ) )) $ \\
	$  \JustPropEQ{\text{Elimination der Implikation}}  \lnot( \lnot (\exists x P_1(x) \rightarrow \lnot P_2(x) ) \lor ( \exists y \lnot ( P_2(y) \land P_1(y) )) ) $\\
	$ \JustPropEQ{\text{DeMorgansche Regel}} \lnot \lnot (\exists x P_1(x) \rightarrow \lnot P_2(x) ) \land \lnot ( \exists y \lnot ( P_2(y) \land P_1(y) ))  $\\
	$ \JustPropEQ{\text{Negierter Existenzquantor}} \lnot \lnot (\exists x P_1(x) \rightarrow \lnot P_2(x) ) \land ( \forall y \lnot \lnot ( P_2(y) \land P_1(y) ))  $\\
	$ \JustPropEQ{\text{Doppelte Negation}} (\exists x P_1(x) \rightarrow \lnot P_2(x) ) \land ( \forall y  P_2(y) \land P_1(y) )  $
	\item[] \hfill \textbf{2 Punkte}
	\item[1c)] %$ (\exists x P_1(x) \rightarrow \lnot P_2(x) ) \rightarrow ( \exists y \lnot ( P_2(y) %\land P_1(y) )  $\\
	Widerspruchsannahme:\\
	$(\exists x P_1(x) \rightarrow \lnot P_2(x) ) \land ( \forall y  P_2(y) \land P_1(y) )  $\\
	(Z1): $\propFalsum$\\
	(A1): $\exists x P_1(x) \rightarrow \lnot P_2(x) $\\
	(A2): $\forall y  P_2(y) \land P_1(y) $\\
	Sei x beliebig aber fest.\\
	(A3): $P_1(x) \rightarrow \lnot P_2(x) $\\
	Waehle $y \deff x$ in A2.\\
	(A4): $P_2(x) \land P_1(x) $\\
	Aus A4 folgen A5 und A6.\\
	(A5): $P_2(x)$\\
	(A6): $P_1(x)$\\
	Aus A6 und A3 folgt A7.\\
	(A7): $\lnot P_2(x)$\\
	Aus A7 und A5 folgt Z1(Widerspruch).\\
	Also gilt $(\exists x P_1(x) \rightarrow \lnot P_2(x) ) \rightarrow ( \exists y \lnot ( P_2(y) \land P_1(y) ))$.
	\item[] \hfill \textbf{5.5 Punkte}
	\item[1d)] %$ (\exists x P_1(x) \rightarrow \lnot P_2(x) ) \rightarrow ( \exists y \lnot ( P_2(y) %\land P_1(y) )  $\\
	Kontraposition:\\
	$ \lnot ( \exists y \lnot ( P_2(y) \land P_1(y) ) \rightarrow \lnot (\exists x P_1(x) \rightarrow \lnot P_2(x) )  $
	\item[] \hfill \textbf{1 Punkte}
\end{compactenum}

\newpage
%%%%%%%%
% Lösungen zur Aufgabe 2
%%%%%%%%

\textbf{Aufgabe 2: Relationen} \hfill \textbf{16 Punkte}
\begin{compactenum}
	\item[2a)] $R_1$ ist nicht rechtstotal, da $1 \in B$ aber $1 \notin R_1$.\\
	$R_1$ ist nicht linkseindeutig, da $aR_14$ und $bR_14$ gelten mit $a \neq b$.\\
	$R_1$ ist nicht rechtseindeutig, da $bR_14$ und $bR_15$ gelten mit $4 \neq 5$.\\
	$R_2$ ist nicht linkstotal, da $4 \in B$ aber $4 \notin R_2$.\\
	$R_2$ ist nicht linkseindeutig, da $1R_2a$ und $5R_2a$ gelten mit $1 \neq 5$.\\
	$R_2$ ist nicht rechtseindeutig, da $3R_2b$ und $3R_2c$ gelten mit $b \neq c$.\\ 
	$R_3$ ist nicht rechtseindeutig, da $cR_3b$ und $cR_3c$ gelten mit $b \neq c$.\\ 
	\item[] \hfill \textbf{3.5 Punkte}
	\item[2b)] $R_4$ ist keine totale Ordnung, da $R_4$ nicht linear ist.\\
	fuer lineare Relationen gilt:\\
	$\AllQ{a,b \in A}{a \neq b \rightarrow (aRb \lor bRa)}$\\
	es gilt $a,f \in D$ mit $a \neq f$, jedoch gelten weder $aR_4f$ noch $fR_4a$.\\
	$R_4$ ist reflexiv, da fuer $R_4$ gilt:\\
	$\AllQ{d \in D}{dR_4d}$\\
	$R_4$ ist transitiv, da fuer $R_4$ gilt:\\
	$\AllQ{a,b,c \in D}{(aR_4b \land bR_4c) \rightarrow aR_4c}$\\
	$R_4$ ist antisymmetrisch, da fuer $R_4$ gilt:\\
	$\AllQ{a,b \in D}{(aR_4b \land bR_4a) \rightarrow a = b}$\\
	Somit gilt: $R_4$ ist eine partielle Ordnung.
	\item[] \hfill \textbf{4.5 Punkte}
	\item[2c)]\textbf{Aequivalenzrelation:} 
	G ist keine Aequivalenzrelation, da die Relation nicht symmetrisch ist. \\
	Es gilt: G ist symmetrisch, falls $\AllQ{x,y \in \mathbb{Z}}{xRy \rightarrow yRx} $\\
	Gegenbeispiel: Wähle x = 5 und y = 10. \\
	Dann ist $(x,y)\in G.$ \\
	Fuer $(y,x)$ gilt: $5=10+n$ mit $n=-5$ \\ 
	Aber: $n \notin \mathbb{N}$, was einen Widerspruch zur Definition von G darstellt. \\
	Somit gilt: G ist nicht symmetrisch. \\
	\textbf{partielle Ordnung:}
	Damit G eine partielle Ordnung ist, muss Reflexivitaet, Antisymmetrie und Transitivitaet gezeigt werden. \\ 
	\begin{itemize}
	    \item Reflexivitaet: \\
	    Zu zeigen (Z1): $\AllQ{x \in \mathbb{Z}}{(x,x)\in G}$ \\
	    (A1): Sei $x\in \mathbb{Z}$ in (Z1). \\
	    Zu zeigen (Z2): $(x,x) \in G$\\
	    $(x,x) \in G \JustAequ{G} \exists n\in \mathbb{N}.\ x=x+n$.\\
	    Gilt $\forall x\in\mathbb{Z}$ mit $n=0$. Damit ist G reflexiv.  
	    \item Antisymmetrie: \\ 
	    Zu zeigen (Z1): $\AllQ{x,y \in \mathbb{Z}}{(x,y)\in G \land (y,x)\in G \rightarrow x=y}$ \\
	    Sei $x,y\in \mathbb{Z}$\\
	    (Z2): $(x,y)\in G \land (y,x)\in G \rightarrow x=y$\\
	    (A1): $(x,y)\in G \land (y,x)\in G$\\
	    (A1.1): $(x,y)\in G$\\
	    (A1.2): $(y,x)\in G$\\
	    (Z3): $a=b$\\
	    \begin{align*}
	        (A1.1)\ (x,y)\in G \JustImp{G}\exists n_1\in \mathbb{N}.\ x = y+n_1  \\
	        (A1.2)\ (y,x)\in G \JustImp{G}\exists n_2\in \mathbb{N}.\ y = x+n_2 \\
	        \JustImp{Gleichheit\ von\ y} \exists n_1,n_2\in \mathbb{N}. x = (x+n_2) + n_1\\ 
	        \JustEq{Assoz.\ der\ Addition}\exists n_1,n_2\in \mathbb{N}. x=x+(n_1+n_2) \\
	    \end{align*}
	    Da es sich bei $n_1$ und $n_2$ um natürliche und somit nicht-negative Zahlen handelt, kann diese Gleichung nur für $n_1=n_2=0$ gelten.\\
	    Einsetzen in (A1.1): $\exists n_1\in \mathbb{N}.\ x = y+n_1 \JustImp{n_1=0} x=y+0 \JustImp{Addition} x=y$\\
	    Somit ist G antisymmetrisch. 
	    \item Transitivitaet \\
	    Zu zeigen (Z1): $\AllQ{x,y,z \in \mathbb{Z}}{(x,y)\in G \land (y,z)\in G \rightarrow (x,z)\in G}$\\
	    Sei $x,y,z \in \mathbb{Z}$\\
	    (Z2): $(x,y)\in G \land (y,z)\in G \rightarrow (x,z)\in G$\\
	    (A1): $(x,y)\in G \land (y,z)\in G$\\
	    (A1.1): $(x,y)\in G$\\
	    (A1.2): $(y,z)\in G$\\
	    (Z3): $(x,z)\in G$\\
	    \begin{align*}
	        (A1.1) (x,y)\in G \JustImp{G} \exists n_1\in \mathbb{N}.\ x = y+n_1 \\
	        (A1.2) (y,z)\in G \justImp{G} \exists n_2\in \mathbb{N}.\ y = z+n_2 \\
	        \JustImp{Gleichheit\ von\ y} \exists n_1,n_2\in \mathbb{N}.\ x=(z+n_2)+n_1 \\
	        \JustEq{Assoz. der Addition} \exists n_1,n_2\in \mathbb{N}.\ x=z+(n_1+n_2)  \\
	    \end{align*}
	    Da die Summe von zwei natürlichen Zahlen immer eine natürliche Zahl ist, gilt nach Def. von G: 
	    $(x,z)\in G$. Somit ist G transitiv. 
	\end{itemize}
	Somit gilt: G ist eine partielle Ordnung. 
	
	\item[] \hfill \textbf{8 Punkte}
\end{compactenum}

\newpage

%%%%%%%%
% Lösungen zur Aufgabe 3
%%%%%%%%

\textbf{Aufgabe 3: Kardinalität} \hfill \textbf{8 Punkte}
\begin{compactenum}
	\item[] Behauptung: Wir geben eine Bijektion $f: M \rightarrow \N$ an.\\
	$x \mapsto \dfrac{x}{41} - 3$\\
	Wir geben eine weitere Funktion $g: \N \rightarrow M$ an.\\
	$ x \mapsto 41x +123$\\
	(Z1): Bijektion($f$)\\
	Wenn $f \circ g = \DIAREL{\N}$ und $g \circ f = \DIAREL{M}$, dann ist laut FS 0.7.8 $f$ eine Bijektion.\\
	(Z1.1): $\AllQ{x \in \N}{f \circ g(x) = \DIAREL{\N}}$\\
	Sei $x \in \N$ beliebig aber fest.\\
	$f \circ g(x) \JustEq{\circ} f(g(x)) \JustEq{g} f(41x + 123) \JustEq{f} \dfrac{41x +123}{41} - 3 = x + 3 - 3= x \JustEq{\Delta} \DIAREL{\N}$\\
	(Z2.1): $\AllQ{x \in M}{g \circ f(x) = \DIAREL{M}}$\\
	Sei $x \in M$ beliebig aber fest.\\
	$g \circ f(x) \JustEq{\circ} g(f(x)) \JustEq{f} g(\dfrac{x}{41} -3) \JustEq{g} 41(\dfrac{x}{41} - 3) + 123 = x - 123 +123= x \JustEq{\Delta} \DIAREL{M}$\\
	Da wir Z1.1 und Z2.1 gezeigt haben, gilt: $f$ ist eine Bijektion. Somit gilt die Aussage.
	
\end{compactenum}

\end{document}


