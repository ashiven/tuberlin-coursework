\documentclass[twoside,10pt,fleqn,headinclude=false]{scrartcl}

\usepackage[utf8]{inputenc}
\usepackage[T1]{fontenc}
\usepackage[ngerman]{babel}
\usepackage{csquotes}
\usepackage{xifthen}
\usepackage{textcomp, xspace, url, amsmath, booktabs, amsfonts, amssymb, version, ntheorem, lastpage, scrtime, pifont, wasysym, tabularx, semantic}
\usepackage{pgf}
\usepackage{tikz}
	\usetikzlibrary{trees, decorations, arrows, automata, shadows, positioning, plotmarks, calc, matrix}
\usepackage[arrow, matrix, curve]{xy}
%\usepackage{scrpage2}
\usepackage[xspace]{ellipsis}
\usepackage{palatino}
\usepackage{ragged2e}
\usepackage[tracking=true]{microtype}
\usepackage{multirow}
\usepackage{paralist}
\usepackage{parskip}
\usepackage[footskip=1cm,bindingoffset=2cm,hmargin={0cm, 0cm}, inner=1cm, outer=3cm, top=2cm, bottom=2cm]{geometry}
	\setlength{\marginparwidth}{2cm}
	\renewcommand{\arraystretch}{1.2}
	\setlength{\arrayrulewidth}{1pt}

\usepackage{ForSA.Macros}

%%%%%%%%%%%%%%%
%  meta data  %
%%%%%%%%%%%%%%%

\author{Edgar Arndt}
\def\xdate{SoSe 2022}

%%%%%%%%%%%%%%%%%%%%%%%%%%
%  counter declarations  %
%%%%%%%%%%%%%%%%%%%%%%%%%%

\newcounter{firstTask}
\newcounter{lastTask}
\newcount\member


% Nummer der ersten Aufgabe in dieser HA-Abgabe
\setcounter{firstTask}{01}
% Nummer der letzten Aufgabe in dieser HA-Abgabe
\setcounter{lastTask}{04}
% durch Kommata getrennte Liste aller Punktzahlen mit einem Komma am Ende
\def\xpoints{19,14,17,10,}

\begin{document}
\pagestyle{empty}

%%%%%%%%%%%%
%  header  %
%%%%%%%%%%%%

\centering\normalfont\Large{\textsc{Formale Sprachen und Automaten}}\\
\raggedright\normalsize{\textsl{MTV: Modelle und Theorie Verteilter Systeme}\hfill\textsl{\xdate}}\\
\centering\Large{Hausaufgabe}\\
\vspace{0.5\baselineskip}\centering\large{Name: <NAME1>}\\
\vspace{0.5\baselineskip}\centering\large{Matrikelnummer: <MATRIKELNUMMER1>}\\
\vspace{0.5\baselineskip}\centering\large{(optional) Name: <NAME2>}\\
\vspace{0.5\baselineskip}\centering\large{(optional) Matrikelnummer: <MATRIKELNUMMER2>}


\newlength{\MaxLength}
	\settowidth{\MaxLength}{\textsc{Korrektur}}
\newcount\taskscount
	\taskscount=\value{lastTask}
	\advance\taskscount by -\value{firstTask}
	\advance\taskscount by 1
\newcount\columncount
	\columncount=\taskscount
	\advance\columncount by 2
\newlength{\itemwidth}
	\settowidth{\itemwidth}{1000}

\def\tabletasks{}{
	\member=\value{firstTask}
	\advance\member by -1
	\loop
		\ifnum\member < \value{lastTask}
		\advance\member by 1
		\appto\tabletasks{& \centering}
		\xappto\tabletasks{\the\member}
	\repeat
}

\newcounter{pointsum}
\setcounter{pointsum}{0}
\def\tablepoints#1,#2\END{
	& \centering \ #1
	\addtocounter{pointsum}{#1}
	\ifx\relax#2\relax\else
	   \tablepoints #2\END
	\fi
}

\def\tablecells{}{
	\member=\value{firstTask}
	\advance\member by -1
	\loop
		\ifnum\member < \value{lastTask}
		\advance\member by 1
		\xappto\tablecells{&}
	\repeat
}


$ $
\vspace{\fill}

\flushleft
Je 3 erreichte Hausaufgabenpunkte entsprechen einem Portfoliopunkt. Es wird mathematisch auf halbe Portfoliopunkte gerundet. 
\vspace{2em}\\
\textbf{Korrektur:} \vspace{0.5em}\\
\begin{tabular}{|>{\scshape}p{\MaxLength}|*{\taskscount}{p{\itemwidth}|}|p{\itemwidth}|}
	\hline
	\textsc{Aufgabe} \tabletasks & {\centering \ $ \sum $}\\
	\hline
	\textsc{Punkte}
	\expandafter\tablepoints \xpoints\END & {\centering \ \arabic{pointsum}}\\
	\hline
	\textsc{Erreicht} \tablecells &\\
	\hline
	\textsc{Korrektur} \tablecells &\\
	\hline
	\multicolumn{\columncount}{|l|}{Portfoliopunkte:}\\
	\hline
\end{tabular}
\vspace{\fill}
\newpage

%%%%%%%
% Eidesstattliche Erklärung
%%%%%%%
\section*{Erklärung über Arbeitsteilung}

Hiermit versichern wir, dass wir alle Aufgaben zu je gleichen Teilen bearbeitet und die vorliegenden Lösungen zu je gleichen Teilen erstellt haben.

\begin{align*}
	& \underline{\hspace{5cm}} && \underline{\hspace{5cm}}\\
	& \text{< dein Nachname1, dein Vorname1 >} && \text{Ort, Datum}
\end{align*}

\begin{align*}
	& \underline{\hspace{5cm}} && \underline{\hspace{5cm}}\\
	& \text{< dein Nachname2, dein Vorname2 >} && \text{Ort, Datum}
\end{align*}

\newpage
%%%%%%%
% Lösungen zur Aufgabe 4
%%%%%%%

\textbf{Aufgabe 4: Pumping Lemma regulärer Sprachen} \hfill \textbf{19 Punkte}
\begin{compactenum}
	\item[4a)] <Lösung>
	\item[] \hfill \textbf{9 Punkte}
	\item[4b)] <Lösung>
	\item[] \hfill \textbf{10 Punkte}

\end{compactenum}

\newpage
%%%%%%%%
% Lösungen zur Aufgabe 5
%%%%%%%%

\textbf{Aufgabe 5: Myhill-Nerode für reguläre Sprachen} \hfill \textbf{14 Punkte}
\begin{compactenum}
	\item[5a)] <Lösung>
	\item[] \hfill \textbf{7 Punkte}
	\item[5b)] <Lösung>
	\item[] \hfill \textbf{7 Punkte}

\end{compactenum}

\newpage

%%%%%%%%
% Lösungen zur Aufgabe 6
%%%%%%%%

\textbf{Aufgabe 6: Myhill-Nerode für nicht-reguläre Sprachen} \hfill \textbf{17 Punkte}
\begin{compactenum}
	\item[] <Lösung>
\end{compactenum}

\newpage
%%%%%%%%
% Lösungen zur Aufgabe 7
%%%%%%%%

\textbf{Aufgabe 7: Beschreiben und erzeugen von regulären Sprachen} \hfill \textbf{10 Punkte}
\begin{compactenum}
	\item[7a)] <Lösung>
	\item[] \hfill \textbf{4 Punkte}
	\item[7b)] <Lösung>
	\item[] \hfill \textbf{6 Punkte}

\end{compactenum}



\end{document}


