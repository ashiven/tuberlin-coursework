% 01.05.2024
% Jannik 
% Novak
% 392210
% nevisha@pm.me

% ---------- PREAMBLE ----------

\documentclass{article}

\usepackage[ngerman]{babel}
\usepackage[utf8]{inputenc}
\usepackage[T1]{fontenc}
\usepackage{blindtext}

\title{Hausaufgabe 2}
\author{Jannik Novak, Ullie Eisenhardt}
\date{09. 05. 2024}

% ---------- TEXT ----------

\begin{document}

\maketitle

\clearpage

\tableofcontents

\clearpage

\section{Auflistungen}

\subsection{Description}

\begin{description}
    \item Hallo
    \item Dido
    \item Moin
\end{description}

\subsection{Itemize}

\begin{itemize}
    \item Hello
    \item Gogo
    \item Yea
    \begin{itemize}
        \item Unterpunkt 1
        \item Unterpunkt 2
    \end{itemize}
\end{itemize}

\subsection{Enumerate}

\begin{enumerate}
    \item Alfons
    \item Bertrecht
    \item Carl
    \begin{enumerate}
        \item Erhardt
        \item Franziska
        \begin{enumerate}
            \item Gunter
            \item Herbert
            \begin{enumerate}
                \item Irine
                \item Juergen
                \item Konrad
            \end{enumerate}
        \end{enumerate}
    \end{enumerate}
\end{enumerate}

\subsection{Itemize und Enumerate}

\begin{itemize}
    \item hallo
    \begin{enumerate}
        \item moin
        \begin{itemize}
            \item servus
            \begin{enumerate}
                \item howdy
                \begin{itemize}
                    \item gruetzi
                    \begin{enumerate}
                        \item nihao
                    \end{enumerate}
                \end{itemize}
            \end{enumerate}
        \end{itemize}
    \end{enumerate}
\end{itemize}

\clearpage

\section{Tabellen}

\subsection{Tabelle 1}

\begin{tabular}{l c c c r }
    left & center & center & center & right \\
    \hline
    1 & 2 & 3 & 4 & 5 \\
    1 & 2 & 3 & 4 & 5 \\
    1 & 2 & 3 & 4 & 5 \\
    1 & 2 & 3 & 4 & 5 \\
\end{tabular}

\subsection{Tabelle 2}

\begin{tabular}{|l|c|c|c|r|}
    left & center & center & center & right \\
    \hline
    \multicolumn{1}{|l}{1} & \multicolumn{1}{c}{2} & \multicolumn{1}{c|}{3} & 4 & 5 \\
    \multicolumn{1}{|l}{1} & \multicolumn{1}{c}{2} & \multicolumn{1}{c|}{3} & 4 & 5 \\
    \multicolumn{1}{|l}{1} & \multicolumn{1}{c}{2} & \multicolumn{1}{c|}{3} & 4 & 5 \\
\end{tabular}

\subsection{Tabelle 3}

\begin{tabular}{*6{|r}|*6{l|}}
    1 & 2 & 3 & 4 & 5 & 6 & 7 & 8 & 9 & 10 & 11 & 12 \\
    \hline
    1 & 2 & 3 & 4 & 5 & 6 & 7 & 8 & 9 & 10 & 11 & 12 \\
    1 & 2 & 3 & 4 & 5 & 6 & 7 & 8 & 9 & 10 & 11 & 12 \\
    1 & 2 & 3 & 4 & 5 & 6 & 7 & 8 & 9 & 10 & 11 & 12 \\
\end{tabular}

\subsection{Tabelle 4}

\begin{tabular}{|l|p{5cm}|r|}
    left & center & right \\
    \hline
    1 & Das hier ist ein etwas langer Text der mehrere Zeilen umfassen wird. Jaja so ist das tatsaechlich. & 3 \\
    1 & Das hier ist ein etwas langer Text der mehrere Zeilen umfassen wird. Jaja so ist das tatsaechlich. & 3 \\
    1 & Das hier ist ein etwas langer Text der mehrere Zeilen umfassen wird. Jaja so ist das tatsaechlich. & 3 \\
    1 & Das hier ist ein etwas langer Text der mehrere Zeilen umfassen wird. Jaja so ist das tatsaechlich. & 3 \\
\end{tabular}

\end{document}
