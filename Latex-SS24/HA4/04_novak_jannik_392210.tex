% 01.05.2024
% Jannik 
% Novak
% 392210
% nevisha@pm.me

% ---------- PREAMBLE ----------

\documentclass{article}

\usepackage[ngerman]{babel}
\usepackage[utf8]{inputenc}
\usepackage[T1]{fontenc}
\usepackage{blindtext}
\usepackage{amsmath}
\usepackage{amssymb}

\title{Hausaufgabe 4}
\author{Jannik Novak, Ullie Eisenhardt}
\date{18. 05. 2024}

\newcommand{\N}{\mathbb{N}}
\newcommand{\Z}{\mathbb{Z}}
\newcommand{\R}{\mathbb{R}}
\newcommand{\C}{\mathbb{C}}

\newcommand{\binomial}[2]{(#1 + #2)^2}
\newcommand{\hallo}[1]{Hallo #1,}

% ---------- TEXT ----------

\begin{document}

\maketitle

\clearpage

\tableofcontents

\clearpage

\section{Aufgabe 4.1}

\subsection{Zahlenmengen}

Die natürlichen Zahlen $\N$, sind eine Untermenge der ganzen Zahlen $\Z$. Diese sind eine Untermenge der reellen Zahlen $\R$. Diese sind wiederum eine Untermenge der komplexen Zahlen $\C$.

\subsection{Kommandos}

\hallo{Peter} wusstest du, dass $\binomial{a}{b} = a^2 + 2ab + b^2$ gilt? 

\section{Aufgabe 4.2}

\subsection{Rand von Mengen}

Es gilt:
%
\begin{align*}
    M_1 & = \{(x,y) \in \R^2 \, | \, |x| + |y| = 1 \} \\
    \mathring{M_1} & = \emptyset \\
    \partial M_1 & = \{(x,y) \in \R^2 \, | \, |x| + |y| = 1 \} = M_1
\end{align*}
%

\subsection{Gradient}

Mit Hilfe der Rechenregeln für Summen (Linearität) erhalten wir:
%
\begin{align*}
    \nabla (f + g)(x) & = \left( \begin{array}{c} \partial_{x_1}(f(x) + g(x)) \\ \cdot \cdot \cdot \\ \partial_{x_n}(f(x) + g(x)) \end{array} \right) \\
    & = \left( \begin{array}{c} \partial_{x_1}f(x) + \partial_{x_1}g(x) \\ \cdot \cdot \cdot \\ \partial_{x_n}f(x) + \partial_{x_n}g(x) \end{array} \right) \\
    & = \nabla f(x) + \nabla g(x)
\end{align*}
%
\clearpage

\subsection{Lineare Gleichungssysteme}

Wir betrachten das folgende lineare Gleichungssytem:
%
\begin{equation*}
    \left( 
    \begin{array}{ccc}
         1 & 1 & 1 \\
         2 & -2 & -1 \\
         4 & 4 & 1
    \end{array} 
    \right) 
    \left( 
    \begin{array}{c}
         c_1 \\
         c_2 \\
         c_3
    \end{array} 
    \right) 
    = 
    \left( 
    \begin{array}{c}
         2 \\
         -1 \\
         0
    \end{array} 
    \right) 
\end{equation*}
%
Wir lösen es mit Hilfe des Gauss-Algorithmus:
%
\begin{align*}
    \begin{array}{ccccc}
         1 & 1 & 1 & | & 2 \\
         2 & -2 & -1 & | & -1 \\
         4 & 4 & 1 & | & 0 
    \end{array}
    & \rightsquigarrow
    \begin{array}{ccccc}
         1 & 1 & 1 & | & 2 \\
         2 & -2 & -1 & | & -1 \\
         0 & 0 & -3 & | & -8 
    \end{array} \\
    & \rightsquigarrow
    \begin{array}{ccccc}
         1 & 1 & 1 & | & 2 \\
         2 & -2 & -1 & | & -1 \\
         0 & 0 & -3 & | & -8 
    \end{array} \\
    & \rightsquigarrow
    \begin{array}{ccccc}
         1 & 1 & 1 & | & 2 \\
         0 & -4 & -3 & | & -5 \\
         0 & 0 & -3 & | & -8 
    \end{array} 
\end{align*}
%

\section{Aufgabe 4.3}

\subsection{Funktion}

Die Funktion
%
\begin{equation*}
    f: \R^2 \rightarrow \R, \qquad f(x,y) = \left\{ \begin{array}{ll}
         x \sin ({1 \over y}) + y \cos ({1 \over x}) & \text{falls} \, x \neq 0 \, \text{und} \, y \neq 0 \\
         0 & \text{falls} \, x = 0 \, \text{und} \, y = 0
    \end{array} \right.
\end{equation*}
%
ist stetig auf $\R^2$ außer auf $\{(x,y) \in \R^2 \, | \, x = 0$ oder $y = 0\} \, \backslash \, \{(0,0)\}$.

\subsection{Kettenbruch}

Mit Hilfe des \texttt{\textbackslash cfrac}-Befehls setzen wir einen schönen Kettenbruch:
%
\begin{equation*}
    {4 \over \pi} = 1 + \cfrac{1^2}{3 + \cfrac{2^2}{5 + \cfrac{3^2}{7 + \cfrac{4^2}{9+\cfrac{5^2}{\ddots}}}}}
\end{equation*}
%

\clearpage

\subsection{Laplace Operator}

Tip: \texttt{split}-Umgebung: \\
Die folgende Rechnung bestätigt unsere Formel für den Laplace-Operator in Polarkoordinaten.
%
\begin{equation}
\begin{split}
    \partial_{r}^{2}g + {1 \over r} \cdot \partial_r g + {1 \over r^2} \cdot \partial_{\varphi}^{2}g & = \cos(\varphi)^2\partial_{x}^{2}f + 2\cos(\varphi)\sin(\varphi)\partial_x \partial_y f \\
    & \quad +\sin(\varphi)^2 \partial_{y}^{2} f + {1 \over r} (\cos(\varphi) \partial_x f + \sin(\varphi) \partial_y f) \\
    & \quad + {1 \over r^2} (-r(\cos(\varphi) \partial_x f + \sin(\varphi) \partial_y f) \\
    & \quad + r^2 (\sin(\varphi)^2 \partial_{x}^{2} f + \cos(\varphi)^2 \partial_{y}^{2} f) \\
    & \quad - 2r^2 \sin(\varphi) \cos(\varphi) \partial_x \partial_y f) \\
    & = \partial_{x}^{2} f ( \cos(\varphi)^2 + \sin(\varphi)^2) \\
    & \quad + \partial_{y}^{2} f (\sin(\varphi)^2 + \cos(\varphi)^2) \\
    & \quad + \partial_x \partial_y f (2\cos(\varphi)sin(\varphi) - 2\sin(\varphi)\cos(\varphi)) \\
    & \quad + \partial_x f ({1 \over r} \cos(\varphi) - {1 \over r}\cos(\varphi)) \\
    & \quad + \partial_y f ({1 \over r} \sin(\varphi) - {1 \over r}\sin(\varphi)) \\
    & = \partial_{x}^{2} f + \partial_{y}^{2} f
\end{split}
\end{equation}
%
\subsection{Maxwellsche Gleichungen}

Wir betrachten die Maxwellschen Gleichungen, wobei wir \textbf{\textit{B}}, \textbf{\textit{E}}, \textbf{\textit{j}} \textit{fett} setzen,
statt sie mit Vektorpfeilen zu versehen, wie es in der Literatur auch vorkommt.
Die Gleichungen sollen zentriert gesetzt werden, also nicht nach dem Gleichheitszeichen ausgerichtet werden.
%
\begin{gather*}
    \nabla \cdot \textbf{\textit{E}} = {\rho \over \varepsilon_0} \\
    \nabla \cdot \textbf{\textit{B}} = 0 \\
    \nabla \times \textbf{\textit{E}} = - {\partial \textbf{\textit{B}} \over \partial t} \\
    \nabla \times \textbf{\textit{B}} = \mu_0 \textbf{\textit{j}} + \mu_0 \varepsilon_0 {\partial \textbf{\textit{E}} \over \partial t}
\end{gather*}
%
\end{document}
