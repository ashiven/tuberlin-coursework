% 01.05.2024
% Jannik 
% Novak
% 392210
% nevisha@pm.me

% ---------- PREAMBLE ----------

\documentclass{article}

\usepackage[ngerman]{babel}
\usepackage[utf8]{inputenc}
\usepackage[T1]{fontenc}
\usepackage{blindtext}
\usepackage{amsmath}
\usepackage{amssymb}
\usepackage{amsthm}
\usepackage{cancel}
\usepackage{mathtools}

\title{Hausaufgabe 5}
\author{Jannik Novak, Ullie Eisenhardt}
\date{26. 05. 2024}

\newtheorem{th1}{Theorem 1}[section]
\newtheorem{th2}{Theorem 2}
\newtheorem{th3}[th2]{Theorem 3}

% ---------- TEXT ----------

\begin{document}

\maketitle

\clearpage

\tableofcontents

\clearpage

\section{Aufgabe 5.1 1)}

\begin{th1}
Das ist ein spannendes Theorem.
\end{th1}

\begin{th2}
Ein Theorem? 
\end{th2}

\begin{th3}
Noch ein Theorem?
\end{th3}

\section{Aufgabe 5.1 2)}

\begin{th1}
Das ist ein spannendes Theorem.
\end{th1}

\begin{th3}
Noch ein Theorem?
\end{th3}

\begin{th2}
Ein Theorem? 
\end{th2}

\section{Aufgabe 5.2}

\subsection{Cancel}

%
\begin{equation*}
    \cancel{ \dfrac{\varphi}{2} }  + 2 = A
\end{equation*}
%

\subsection{Matrizen}

%
\begin{equation*}
    \left(
    \begin{array}{ccc}
         1 & 2 & 3\\
         10 & 20 & 30 \\
         100 & 200 & 300 
    \end{array}
    \right)
    = 
    \left(
    \begin{array}{lll}
         1 & 2 & 3\\
         10 & 20 & 30 \\
         100 & 200 & 300 
    \end{array}
    \right)
    = 
    \left(
    \begin{array}{rrr}
         1 & 2 & 3\\
         10 & 20 & 30 \\
         100 & 200 & 300 
    \end{array}
    \right)
\end{equation*}
%

\subsection{Intertext}

%
\begin{align*}
    A = \sum_{0 \leq i < j \leq n} a_{ij}
\intertext{sieht hässlich aus im Vergleich zu}
    A = \sum_{\mathclap{0 \leq i < j \leq n}} a_{ij}
\end{align*}
%

\clearpage

\section{Aufgabe 5.3 1)}

\subsection{Formeln}

%
\begin{equation}
\label{eq:py}
    a^2+b^2 = c^2
\end{equation}
%

%
\begin{equation}
\label{eq:bim}
    (a-b)^2 = a^2 -2ab + b^2
\end{equation}
%

%
\begin{equation}
\label{eq:bip}
    (a+b)^2 = a^2 +2ab +b^2
\end{equation}
%

\section{Aufgabe 5.3 2)}
\label{sec:0}

Dies ist eine Referenz zur ersten Gleichung~\eqref{eq:py}. Dies ist eine Referenz zur zweiten Gleichung~\eqref{eq:bim}. Die letzte Gleichung~\eqref{eq:bip}.

\subsection{Ebene 1}
\label{sec:1}

Hier beziehen wir uns auf den Abschnitt~\ref{sec:0}. Noch eine Referenz zum Abschnitt~\ref{sec:2}.

\subsubsection{Ebene 2}
\label{sec:2}

Hier eine Referenz zum Abschnitt~\ref{sec:1}. Weiterhin ein Referenz zur Seite \pageref{sec:0}.

\end{document}
