% 01.05.2024
% Jannik 
% Novak
% 392210
% nevisha@pm.me

% ---------- PREAMBLE ----------

\documentclass{scrartcl}

\usepackage[ngerman]{babel}
\usepackage[utf8]{inputenc}
\usepackage[T1]{fontenc}
\usepackage{blindtext}
\usepackage{amsmath}
\usepackage{amssymb}
\usepackage{amsthm}
\usepackage{cancel}
\usepackage{mathtools}
\usepackage{graphicx}
\usepackage{tikz}

\usetikzlibrary{matrix}

\title{Hausaufgabe 6}
\author{Jannik Novak, Ullie Eisenhardt}
\date{05. 06. 2024}

% ---------- TEXT ----------

\begin{document}

\maketitle

\clearpage

\tableofcontents

\clearpage

\section{Aufgabe 6.1}

Dies ist ein sehr sehr sehr langer langer laaaaaaanger langer Text der sehr lang ist und weiterhin ist der Text lang und verweist auf die Tabelle~\ref{tab:table1}. Der Text besteht auch noch aus einem weiteren Satz weil er so lang ist.\\

%
\begin{table}[h]
    \centering
    \captionabove{Die erste Tabelle}
    \label{tab:table1}
    \begin{tabular}{c|c}
         Moin & Tach \\
         \hline
         Hallo & Hallo  \\
         Hey & Hey 
    \end{tabular}
\end{table}
%

\section{Aufgabe 6.2}

Dies ist ein Text der einerseits auf die Abbildung~\ref{fig:bild1} verweist und weiterhin auf die Abbildung~\ref{fig:bild23} verweist.

%
\begin{figure}[h]
    \centering
    \includegraphics[width=0.9\textwidth]{bild1.jpg}
    \caption{Herr Knuth}
    \label{fig:bild1}
\end{figure}
%

%
\begin{figure}[h]
    \centering
    \includegraphics[width=0.45\textwidth]{bild2.jpg}
    \includegraphics[width=0.45\textwidth]{bild3.png}
    \caption{Statue und Filmband}
    \label{fig:bild23}
\end{figure}
%

\section{Aufgabe 6.3}

Dies ist eine Referenz zur Abbildung~\ref{fig:tik1} und dies ist eine weitere Referenz zur Abbildung~\ref{fig:tik2}.\\

%
\begin{figure}[h]
    \begin{tikzpicture}[domain=0:4]
        \draw[very thin,color=gray] (-0.1,-1.1) grid (3.9,3.9);
        \draw[->] (-0.2,0) -- (4.2,0) node[right] {$x$};
        \draw[->] (0,-1.2) -- (0,4.2) node[above] {$f(x)$};
        \draw[color=red] plot[id=x] function{x} 
            node[right] {$f(x) =x$};
        \draw[color=blue] plot[id=sin] function{sin(x)} 
            node[right] {$f(x) = \sin x$};
        \draw[color=orange] plot[id=exp] function{0.05*exp(x)} 
            node[right] {$f(x) = \frac{1}{20} \mathrm e^x$};
    \end{tikzpicture}
    \caption{Beispiel}
    \label{fig:tik1}
\end{figure}
%

%
\begin{figure}[h]
    \begin{tikzpicture}
      \matrix (m) [matrix of math nodes, row sep=3em,
        column sep=3em]{
        & f^\ast E_V& & \vphantom{f^\ast}E_V \\
        f^\ast E & & \vphantom{f^\ast}E & \\
        & U & & V \\
        M & & N & \\};
      \path[-stealth]
        (m-1-2) edge (m-1-4) edge (m-2-1)
                edge [densely dotted] (m-3-2)
        (m-1-4) edge (m-3-4) edge (m-2-3)
        (m-2-1) edge [-,line width=6pt,draw=white] (m-2-3)
                edge (m-2-3) edge (m-4-1)
        (m-3-2) edge [densely dotted] (m-3-4)
                edge [densely dotted] (m-4-1)
        (m-4-1) edge (m-4-3)
        (m-3-4) edge (m-4-3)
        (m-2-3) edge [-,line width=6pt,draw=white] (m-4-3)
                edge (m-4-3);
    \end{tikzpicture}
    \caption{Diagramm}
    \label{fig:tik2}
\end{figure}
%

\end{document}
