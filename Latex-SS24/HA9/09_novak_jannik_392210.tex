% 24.06.2024
% Jannik 
% Novak
% 392210
% nevisha@pm.me

% ---------- PREAMBLE ----------

\documentclass{beamer}

\usepackage[ngerman]{babel}
\usepackage[utf8]{inputenc}
\usepackage[T1]{fontenc}
\usepackage{blindtext}
\usepackage{graphicx}
\usepackage{booktabs}
\usepackage{subfigure}

\mode<presentation>{
\usetheme{Frankfurt}
\usecolortheme{seahorse}
\useinnertheme{circles}
\useoutertheme{infolines}
}

\title{Hausaufgabe 9}
\author{Jannik Novak, Ullie Eisenhardt}
\date{24. 06. 2024}

% ---------- TEXT ----------

\begin{document}

\begin{frame}
\maketitle
\end{frame}

\begin{frame}
\frametitle{Inhalt}
\tableofcontents
\end{frame}

\section{Aufgabe 9.1 a)}
\subsection{Folie 1}
\begin{frame}
\frametitle{Folie 1}
\blindtext
\end{frame}

\subsection{Folie 2}
\begin{frame}
\frametitle{Folie 2}
\blindtext
\end{frame}

\section{Aufgabe 9.1 b)}
\subsection{Folie 1}
\begin{frame}
\frametitle{Folie 1}
\blindtext
\end{frame}

\section{Aufgabe 9.2}
\begin{frame}
\frametitle{Spalten}
\begin{columns}[c]
\column{.5\textwidth}
\textbf{Dinge}
\begin{enumerate}
    \item Eins
    \item Zwei
    \item Drei
\end{enumerate}
\end{columns}    
\end{frame}

\begin{frame}
\frametitle{Tabelle}
\begin{table}[]
    \centering
    \begin{tabular}{c|c}
        Hallo & Hi \\
        \hline \\
        Guten & Morgen \\
        Guten & Abend
    \end{tabular}
\end{table}
\end{frame}

\begin{frame}
\frametitle{Theorem}
\begin{theorem}
    $a^2 + b^2 = c^2$
\end{theorem}
\end{frame}

\begin{frame}
\frametitle{Bilder}
\begin{figure}[t!]
    \centering
    \begin{subfigure}
        \centering
        \includegraphics[width=0.4\textwidth]{bild1}    
    \end{subfigure}
    \begin{subfigure}
        \centering
        \includegraphics[width=0.4\textwidth]{bild2}
    \end{subfigure}
\end{figure}
\end{frame}

\begin{frame}
\frametitle{Bild mit Text}
    \begin{columns}
        \column{0.5\textwidth}
        \includegraphics[width=0.5\textwidth]{bild3}
        \column{0.5\textwidth}
        \blindtext
    \end{columns}
\end{frame}

\begin{frame}
\frametitle{Formel}
\begin{align*} 
4x - 7y &=  10 \\ 
2x + 7y &=  -13
\end{align*}
\end{frame}

\section{Aufgabe 9.3}
\subsection{Overlay}
\begin{frame}
\frametitle{Overlay}
\begin{itemize}
\item 7 is prime (two divisors: 1 and 7).
\pause
\item 8 is not prime (\alert{four} divisors: 1, 2, 4, and 8).
\pause
\item 9 is not prime (\alert{three} divisors: 1, 3, and 9).
\end{itemize}
\end{frame}

\end{document}
