\documentclass[10pt,a4paper]{scrartcl}

%%% Eine Erklärung der meisten eingebundenen Pakete finden sie in der LaTeX-Vorlage für die freiwilligen Hausaufgaben.
\usepackage[utf8]{inputenc}
\usepackage{fullpage}
\usepackage[ngerman]{babel}
\usepackage{dsfont}
\usepackage{hyperref}
\hypersetup{
	colorlinks=true,
	linkcolor=black!20!blue,
	citecolor=black!20!blue,
	urlcolor=black!20!blue
}
\usepackage{booktabs}
\usepackage{stmaryrd}
\usepackage{amsmath}
\usepackage{amssymb}
\usepackage{ebproof}
\usepackage{tikz}
\usepackage{algorithm}
\usepackage{forloop}
\usepackage{MnSymbol}
\usepackage[noend]{algpseudocode}

\usetikzlibrary{calc,through,arrows,automata,decorations.pathmorphing,decorations.pathreplacing,shapes,backgrounds,fit,positioning,shapes.geometric,patterns}

\usepackage{xstring}
\usepackage{xspace}

\usepackage{enumitem}
\setlist[enumerate]{label=(\roman*)}


%%% Zusätzliche Befehle %%%
\newcommand{\N}{\ensuremath{\mathds{N}}}
\newcommand{\Z}{\ensuremath{\mathds{Z}}}
\newcommand{\Q}{\ensuremath{\mathds{Q}}}
\newcommand{\R}{\ensuremath{\mathds{R}}}
\newcommand{\AL}{\ensuremath{\mathrm{AL}}\xspace}
\newcommand{\AVAR}{\textsc{AVar}}
\newcommand{\VAR}{\textsc{Var}}
\newcommand{\EF}{Ehrenfeucht-Fra\"{i}ss\'{e}\xspace}
\newcommand*{\FO}[1]{\ifx\relax #1\relax
	\ensuremath{\mathrm{FO}}\xspace
	\else
	\ensuremath{\mathrm{FO}[#1]}\xspace
	\fi}
\newcommand*{\Eval}[2]{\ensuremath{\left\llbracket #1 \right\rrbracket}^{#2}}
\newcommand{\sequent}{\Rightarrow}
\newcommand*{\axiom}[2]{\overline{#1 \overset{\phantom{.}}{\sequent} #2}}
% Makros für Induktionsbeweise
\newenvironment{induktion}
{
	\newif\ifIA
	\newif\ifIV
	\newif\ifIS
	\IAfalse
	\IVfalse
	\ISfalse
}
{
	\ifIA
	\else
	\errmessage{Der Induktionsanfang fehlt}
	\fi
	\ifIV
	\else
	\errmessage{Die Induktionsvoraussetzung fehlt}
	\fi
	\ifIS
	\else
	\errmessage{Der Induktionsschritt fehlt}
	\fi
}
\newcommand{\InduktionsAnfang}[1][]{\IAtrue \noindent\textbf{IA:} #1\\}
\newcommand{\InduktionsVoraussetzung}[1][]{\IVtrue \noindent\textbf{IV:} #1\\}
\newcommand{\InduktionsSchritt}[1][]{\IStrue \noindent\textbf{IS:}
	\ifx\relax #1\relax
	\else
	zu zeigen: #1
	\fi\\}
% Kiste für die Punkte der Teilaufgaben
	\newcounter{Iterator}
	\newcounter{Aux}
\newcommand{\PunkteKiste}[1]{%
	\newcommand{\CellHeight}{1}
	\newcommand{\CellWidth}{1.15}
	\begin{tikzpicture}
		\setcounter{Aux}{#1}
		\stepcounter{Aux}
		\forloop{Iterator}{1}{\value{Iterator} < \value{Aux}}{
			\node (v) at (\CellWidth * \value{Iterator}, \CellHeight+0.3) {({\roman{Iterator}})};%
			\draw[line width=0.5pt] (\CellWidth * \value{Iterator}+0.5, \CellHeight) -- (\CellWidth * \value{Iterator}+0.5,0);
			}
		\node (f) at (\CellWidth * \value{Aux}, \CellHeight+0.3) {Form};
		\draw[line width=0.5pt] (\CellWidth * \value{Aux} + 0.5, 0) -- (\CellWidth * \value{Aux} + 0.5, \CellHeight);
		\stepcounter{Aux}
		\node (s) at (\CellWidth * \value{Aux}, \CellHeight+0.3) {$\sum$};
		\draw[line width=0.5pt] (0.5 * \CellWidth, 0) -- (0.5 * \CellWidth, \CellHeight);
		\draw[line width=0.5pt] (\CellWidth * \value{Aux} + 0.5, 0) -- (\CellWidth * \value{Aux} + 0.5, \CellHeight);
		\draw[line width=0.5pt] (0.5 * \CellWidth, \CellHeight) -- (\CellWidth*\value{Aux}+0.5,\CellHeight);
		\draw[line width=0.5pt] (0.5 * \CellWidth, 0) -- (\CellWidth*\value{Aux}+0.5,0);
	\end{tikzpicture}
	}
% Aufgabe Umgebung
\newcounter{aufgabeNum}
\newenvironment{aufgabe}[1]
{

	\noindent\textbf{\underline{Aufgabe \refstepcounter{aufgabeNum}\arabic{aufgabeNum}}}\hfill\PunkteKiste{#1}

}
{}


%%% ANFANG DECKBLATT %%%

\begin{document}
\noindent
\noindent \large WiSe 21/22 \hfill
Logik

\begin{center}
\textbf{Hausarbeit 1 Aufgabe 1}\\
\vspace*{0.5cm}
\textbf{Gruppe:} 402355, 392210, 413316, 457146  % Tragen Sie hier die Matrikelnummern der Mitglieder ihrer Gruppe ein.
\end{center}

\newpage

%%% ENDE DECKBLATT %%%

\section*{Lösungen}
\begin{aufgabe}{4}
    \begin{enumerate}
    	\item
    	Wir zeigen die Gueltigkeit dieser Aussage mithilfe des Kompaktheitssatzes.\\ 
    	In den Folien der Vorlesung aus Woche 5 steht:\\
    	
    	$"$Sei $C$ eine unendlich unerfüllbare Klauselmenge, dann folgt aus dem Kompaktheitssatz, dass
        bereits eine endliche Teilmenge $C_0 \subseteq C$ unerfüllbar ist. Also hat $C_0$ eine Resolutionswiderlegung. Diese ist aber auch eine Resolutionswiderlegung von $C$.$"$\\
        
        Sei $Z$ eine unendliche Klauselmenge mit $\Phi \subseteq Z$ und eine endliche Teilmenge $\Phi^{'} \subseteq \Phi$.\\ Wenn es eine Resolutionswiderlegung für $\Phi'$ gibt, so gibt es auch eine Resolutionswiderlegung für $\Phi$, da $\Phi' \subseteq \Phi$.
    	\item
    	Aus Teilaufgabe (i) wissen wir, wenn eine Resolutionswiderlegung fuer eine Teilmenge $\Phi' \subseteq \Phi$ existiert, so existiert auch eine Resolutionswiderlegung fuer die Formelmenge $\Phi$.\\
    	Wir definieren also die folgenden Teilmengen:
    	\[
    	\Phi_{H,E}' := \{E_{1,4}, E_{1,8}, E_{4,8}, E_{2,3}, E_{2,6}, E_{3,6}\}, \Phi_{H,E}' \subseteq \Phi_{H,E}
    	\]
    	\[
    	\Phi_{H,2}' := \{\lnot E_{1,4} \lor \lnot E_{1,8}\lor  \lnot E_{4,8} \lor \lnot E_{2,3} \lor  \lnot E_{2,6} \lor \lnot E_{3,6}\}, \Phi_{H,2}' \subseteq \Phi_{H,2}
    	\]
    	\[
    	\Phi_{H}' := \Phi_{H,E}' \cup \Phi_{H,2}', \Phi_{H}' \subseteq \Phi_{H}
    	\]
    	Wir geben im folgenden eine Resolutionswiderlegung fuer $\Phi_{H}'$ an:
    	\\
    	\begin{tikzpicture}[scale=1]
		\node (A0) at (7.5,0) {$\{E_{1,4}\}$};
		\node (A1) at (6,0) {$\{E_{1,8}\}$};
		\node (A2) at (4.5,0) {$\{E_{4,8}\}$};
		\node (A3) at (3,0) {$\{E_{2,3}\}$};
		\node (A4) at (1.5,0) {$\{E_{2,6}\}$};
		\node (A5) at (0,0) {$\{E_{3,6}\}$};
		\node (A6) at (12,0) {$\{\lnot E_{1,4}, \lnot E_{1,8},\lnot E_{4,8},\lnot E_{2,3},\lnot E_{2,6},\lnot E_{3,6}\}$};
		
		% Die ersten abgeleiteten Klauseln, eine Ebene tiefer und leicht versetzt.
		\node (B0) at (11,-1) {$\{\lnot E_{1,8},\lnot E_{4,8},\lnot E_{2,3},\lnot E_{2,6},\lnot E_{3,6}\}$};
		
		\node (C0) at (9,-2) {$\{\lnot E_{4,8},\lnot E_{2,3},\lnot E_{2,6},\lnot E_{3,6}\}$};
		
		\node (D0) at (7,-3) {$\{\lnot E_{2,3},\lnot E_{2,6},\lnot E_{3,6}\}$};

		\node (E0) at (5,-4) {$\{\lnot E_{2,6},\lnot E_{3,6}\}$};
		
		\node (F0) at (3,-5) {$\{\lnot E_{3,6}\}$};

		\node (G0) at (1,-6) {$\Box $};
        
        \path
        (A0) edge (B0)
        (A6) edge (B0)
        (B0) edge (C0)
        (A1) edge (C0)
        (C0) edge (D0)
        (A2) edge (D0)
        (A3) edge (E0)
        (D0) edge (E0)
        (A2) edge (D0)
        (E0) edge (F0)
        (A3) edge (E0)
        (F0) edge (G0)
        (A4) edge (F0)
        (A5) edge (G0)
        ;
	    \end{tikzpicture}
	    Da $\Phi_{H}' \subseteq \Phi_{H}$ gibt es somit auch eine Resolutionswiderlegung fuer $\Phi_{H}$.
    	\item
    	Aus Teilaufgabe (i) wissen wir, wenn eine Resolutionswiderlegung fuer eine Teilmenge $\Phi' \subseteq \Phi$ existiert, so existiert auch eine Resolutionswiderlegung fuer die Formelmenge $\Phi$.\\
    	\\
    	Angenommen es gibt zwei disjunkte Kreise $C$ und $K$ der Laenge drei in $\Phi_{G}$, dann wählen wir eine Teilmenge $\Phi_{G}' = \Phi_{G,E}' \cup \Phi_{G,2}'$, $\Phi^{'}_{G} \subseteq \Phi_{G}$ und betrachten die zwei disjunkten Kreise:
    	\[
    	C = \{E_{a,b}, E_{a,c}, E_{b,c}\}, K = \{E_{u,v}, E_{u,w}, E_{v,w}\}
    	\]
    	\[
    	\vert\{a,b,c,u,v,w\}\vert = 6, 1 \leq a \leq b \leq c \leq \vert V(G)\vert, 1 \leq u \leq v \leq w \leq \vert V(G)\vert
    	\]
    	Wir definieren $\Phi_{G,E}'$ wie folgt:
    	\[
    	\Phi_{G,E}'=\{E_{a,b},E_{a,c},E_{b,c},E_{u,v},E_{u,w},E_{v,w}\}
    	\]
    	%Betrachten wir nun $\Phi^{'}_{G,2}$, dann erkennen wir, dass $\Phi^{'}_{G,2}$ aus negierten Kanten von zwei mengen von 3 %unterschiedlichen knoten besteht. Egal wie also \{a,b,c\} und \{u,v,w\} belegt sind, es werden nur existierende Kanten negiert.\\
    	Weiterhin definieren wir $\Phi_{G,2}'$ wie folgt:
    	\[
    	\Phi_{G,2}' = \{\lnot E_{a,b} \lor \lnot E_{a,c} \lor \lnot E_{b,c} \lor \lnot E_{u,v} \lor \lnot E_{u,w} \lor \lnot E_{v,w}\}
    	\]
    	Im folgenden geben wir eine Resolutionswiderlegung fuer $\Phi_{G}'$ an:
    	%Da in $\Phi^{'}_{G,E}$ alle möglichen Kanten enhalten sind, kann $\Phi^{'}_{G}$ analog zur Teilaufgabe ii) mit dem %Resolutionskalkül gelöst werden und somit existiert auch eine Resolutionswiderlegung für $\Phi_{G}$
    	\\
    	\begin{tikzpicture}[scale=1]
		\node (A0) at (7.5,0) {$\{E_{v,w}\}$};
		\node (A1) at (6,0) {$\{E_{u,w}\}$};
		\node (A2) at (4.5,0) {$\{E_{u,v}\}$};
		\node (A3) at (3,0) {$\{E_{b,c}\}$};
		\node (A4) at (1.5,0) {$\{E_{a,c}\}$};
		\node (A5) at (0,0) {$\{E_{a,b}\}$};
		\node (A6) at (12,0) {$\{\lnot E_{a,b}, \lnot E_{a,c},\lnot E_{b,c},\lnot E_{u,v},\lnot E_{u,w},\lnot E_{v,w}\}$};
		
		% Die ersten abgeleiteten Klauseln, eine Ebene tiefer und leicht versetzt.
		\node (B0) at (11,-1) {$\{\lnot E_{a,b},\lnot E_{a,c},\lnot E_{b,c},\lnot E_{u,v},\lnot E_{u,w}\}$};
		
		\node (C0) at (9,-2) {$\{\lnot E_{a,b},\lnot E_{a,c},\lnot E_{b,c},\lnot E_{u,v}\}$};
		
		\node (D0) at (7,-3) {$\{\lnot E_{a,b},\lnot E_{a,c},\lnot E_{b,c}\}$};

		\node (E0) at (5,-4) {$\{\lnot E_{a,b},\lnot E_{a,c}\}$};
		
		\node (F0) at (3,-5) {$\{\lnot E_{a,b}\}$};

		\node (G0) at (1,-6) {$\Box $};
        
        \path
        (A0) edge (B0)
        (A6) edge (B0)
        (B0) edge (C0)
        (A1) edge (C0)
        (C0) edge (D0)
        (A2) edge (D0)
        (A3) edge (E0)
        (D0) edge (E0)
        (A2) edge (D0)
        (E0) edge (F0)
        (A3) edge (E0)
        (F0) edge (G0)
        (A4) edge (F0)
        (A5) edge (G0)
        ;
	    \end{tikzpicture}
    	Da $\Phi_{G}' \subseteq \Phi_{G}$ gibt es somit auch eine Resolutionswiderlegung fuer $\Phi_{G}$. Somit ist $\Phi_{G}$ unerfuellbar.
    	
    	\item
    	Angenommen $\Phi_{G}$ ist unerfuellbar.\\Die Formelmenge $\Phi_{G}$ ist wie folgt definiert:
    	\[
    	\Phi_{G} = \Phi_{G,E} \cup \Phi_{G,2}
    	\]
    	Die Formelmenge $\Phi_{G,E}$ beinhaltet alle Kanten und Nichtkanten des Graphen $G$ und wird somit fuer $G$ erfuellt.\\
    	
    	Sei die Formel $\varphi$, $\varphi \in \Phi_{G,2}$ wie folgt definiert:
    	\[
    	\varphi = \lnot E_{a,b} \lor \lnot E_{a,c} \lor \lnot E_{b,c} \lor \lnot E_{u,v} \lor \lnot E_{u,w} \lor \lnot E_{v,w}
    	\]
    	\[
    	\vert\{a,b,c,u,v,w\}\vert = 6, 1 \leq a \leq b \leq c \leq \vert V(G)\vert, 1 \leq u \leq v \leq w \leq \vert V(G)\vert
    	\]
    	Es werden alle Formeln $\psi \in \Phi_{G,2}$ durch $\varphi$ allgemein dargestellt.\\

    	Wenn die Formelmenge $\Phi_{G}' := \Phi_{G,E} \cup \{\varphi\}$ unerfuellbar ist, so folgt aus dem Kompaktheitssatz auch $\Phi_{G}$ ist unerfuellbar, da $\Phi_{G}' \subseteq \Phi_{G}$.\\
    	Somit folgt laut unserer Annahme, in $G$ existieren die Kanten:
    	\[
    	\ E_{a,b}, E_{a,c}, E_{b,c}, E_{u,v}, E_{u,w}, E_{v,w}
    	\]
    	,da nur in diesem Fall $\Phi_{G}'$ unerfuellbar ist.\\
    	
    	Um dies zu zeigen definieren wir die Formelmenge $\Phi_{G,E}'$, $\Phi_{G,E}' \subseteq \Phi_{G,E}$ wie folgt:
        \[
    	\Phi_{G,E}' := \{E_{a,b}, E_{a,c}, E_{b,c}, E_{u,v}, E_{u,w}, E_{v,w}\}
    	\]
    	Es existiert eine Resolutionswiderlegung fuer $\Phi_{G}'' := \Phi_{G,E}' \cup \{\varphi\}$, $\Phi_{G}'' \subseteq \Phi_{G}'$ und somit laut Teilaufgabe (i) auch fuer $\Phi_{G}'$. Jedoch offensichtlich nur genau dann wenn $\Phi_{G,E}'$ die besagten Kanten enthaelt. Dies haben wir bereits in der Teilaufgabe (iii) veranschaulicht.\\
    	
        Somit existieren in G die disjunkten Kreise $C$ und $K$ der Laenge drei:
        \[
        C = \{E_{a,b}, E_{a,c}, E_{b,c}\}, K = \{E_{u,v}, E_{u,w}, E_{v,w}\}
        \]
 
        
    \end{enumerate}
\end{aufgabe}

\end{document}
