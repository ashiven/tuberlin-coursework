\documentclass[10pt,a4paper]{scrartcl}

%%% Eine Erklärung der meisten eingebundenen Pakete finden sie in der LaTeX-Vorlage für die freiwilligen Hausaufgaben.
\usepackage[utf8]{inputenc}
\usepackage{fullpage}
\usepackage[ngerman]{babel}
\usepackage{dsfont}
\usepackage{hyperref}
\hypersetup{
	colorlinks=true,
	linkcolor=black!20!blue,
	citecolor=black!20!blue,
	urlcolor=black!20!blue
}
\usepackage{booktabs}
\usepackage{stmaryrd}
\usepackage{amsmath}
\usepackage{amssymb}
\usepackage{ebproof}
\usepackage{tikz}
\usepackage{algorithm}
\usepackage{forloop}
\usepackage{MnSymbol}
\usepackage[noend]{algpseudocode}

\usetikzlibrary{calc,through,arrows,automata,decorations.pathmorphing,decorations.pathreplacing,shapes,backgrounds,fit,positioning,shapes.geometric,patterns}

\usepackage{xstring}
\usepackage{xspace}

\usepackage{enumitem}
\setlist[enumerate]{label=(\roman*)}


%%% Zusätzliche Befehle %%%
\newcommand{\N}{\ensuremath{\mathds{N}}}
\newcommand{\Z}{\ensuremath{\mathds{Z}}}
\newcommand{\Q}{\ensuremath{\mathds{Q}}}
\newcommand{\R}{\ensuremath{\mathds{R}}}
\newcommand{\AL}{\ensuremath{\mathrm{AL}}\xspace}
\newcommand{\AVAR}{\textsc{AVar}}
\newcommand{\VAR}{\textsc{Var}}
\newcommand{\EF}{Ehrenfeucht-Fra\"{i}ss\'{e}\xspace}
\newcommand*{\FO}[1]{\ifx\relax #1\relax
	\ensuremath{\mathrm{FO}}\xspace
	\else
	\ensuremath{\mathrm{FO}[#1]}\xspace
	\fi}
\newcommand*{\Eval}[2]{\ensuremath{\left\llbracket #1 \right\rrbracket}^{#2}}
\newcommand{\sequent}{\Rightarrow}
\newcommand*{\axiom}[2]{\overline{#1 \overset{\phantom{.}}{\sequent} #2}}
% Makros für Induktionsbeweise
\newenvironment{induktion}
{
	\newif\ifIA
	\newif\ifIV
	\newif\ifIS
	\IAfalse
	\IVfalse
	\ISfalse
}
{
	\ifIA
	\else
	\errmessage{Der Induktionsanfang fehlt}
	\fi
	\ifIV
	\else
	\errmessage{Die Induktionsvoraussetzung fehlt}
	\fi
	\ifIS
	\else
	\errmessage{Der Induktionsschritt fehlt}
	\fi
}
\newcommand{\InduktionsAnfang}[1][]{\IAtrue \noindent\textbf{IA:} #1\\}
\newcommand{\InduktionsVoraussetzung}[1][]{\IVtrue \noindent\textbf{IV:} #1\\}
\newcommand{\InduktionsSchritt}[1][]{\IStrue \noindent\textbf{IS:}
	\ifx\relax #1\relax
	\else
	zu zeigen: #1
	\fi\\}
% Kiste für die Punkte der Teilaufgaben
	\newcounter{Iterator}
	\newcounter{Aux}
\newcommand{\PunkteKiste}[1]{%
	\newcommand{\CellHeight}{1}
	\newcommand{\CellWidth}{1.15}
	\begin{tikzpicture}
		\setcounter{Aux}{#1}
		\stepcounter{Aux}
		\forloop{Iterator}{1}{\value{Iterator} < \value{Aux}}{
			\node (v) at (\CellWidth * \value{Iterator}, \CellHeight+0.3) {({\roman{Iterator}})};%
			\draw[line width=0.5pt] (\CellWidth * \value{Iterator}+0.5, \CellHeight) -- (\CellWidth * \value{Iterator}+0.5,0);
			}
		\node (f) at (\CellWidth * \value{Aux}, \CellHeight+0.3) {Form};
		\draw[line width=0.5pt] (\CellWidth * \value{Aux} + 0.5, 0) -- (\CellWidth * \value{Aux} + 0.5, \CellHeight);
		\stepcounter{Aux}
		\node (s) at (\CellWidth * \value{Aux}, \CellHeight+0.3) {$\sum$};
		\draw[line width=0.5pt] (0.5 * \CellWidth, 0) -- (0.5 * \CellWidth, \CellHeight);
		\draw[line width=0.5pt] (\CellWidth * \value{Aux} + 0.5, 0) -- (\CellWidth * \value{Aux} + 0.5, \CellHeight);
		\draw[line width=0.5pt] (0.5 * \CellWidth, \CellHeight) -- (\CellWidth*\value{Aux}+0.5,\CellHeight);
		\draw[line width=0.5pt] (0.5 * \CellWidth, 0) -- (\CellWidth*\value{Aux}+0.5,0);
	\end{tikzpicture}
	}
% Aufgabe Umgebung
\newcounter{aufgabeNum}
\newenvironment{aufgabe}[1]
{

	\noindent\textbf{\underline{Aufgabe \refstepcounter{aufgabeNum}\arabic{aufgabeNum}}}\hfill\PunkteKiste{#1}

}
{}


%%% ANFANG DECKBLATT %%%

\begin{document}
\noindent
\noindent \large WiSe 21/22 \hfill
Logik

\begin{center}
\textbf{Hausarbeit 1 Aufgabe 2}\\
\vspace*{0.5cm}
\textbf{Gruppe:} 402355, 392210, 413316, 457146  % Tragen Sie hier die Matrikelnummern der Mitglieder ihrer Gruppe ein.
\end{center}

\newpage

%%% ENDE DECKBLATT %%%

\section*{Lösungen}
\setcounter{aufgabeNum}{1}
\begin{aufgabe}{2}
    \begin{enumerate}
    	\item
    	$\varphi_{\top} := \neg (\neg X \land X \land \neg X)$, mit $\varphi_{\top} \in$ AL$_1$ und $X \in$ AL$_1$ für alle $X \in \AVAR$
    	
    	\item
    	 Wir zeigen per Induktion, dass für alle $\varphi \in$ AL eine äquivalente Formel $\varphi' \in$ AL$_1$ existiert. 
        
        \begin{induktion}
        \InduktionsAnfang
        Es gilt $X \in$ AL$_1$ für alle $X \in$ \AVAR. \\
        
        Für $\top$ benutzen wir die in Aufgabenteil (i) definierte Formel $\varphi_{\top}\in$ AL$_1$.\\
        Für diese gilt:\\
        $\varphi_{\top} := \neg (\neg X \land X \land \neg X)$ \\
        $\equiv X \lor \neg X \lor X$ \\
        $\equiv X \lor \neg X$ \\
        $\equiv \top$ \\
        
        Für $\bot$ definieren wir die Formel $\varphi_{\bot} \in$ AL$_1$ durch 
        $\varphi_{\bot} := \neg X \land X \land X$. \\
        Für diese gilt:\\
        $\varphi_{\bot} := \neg X \land X \land X$ \\
        $\equiv \neg X \land X$ \\
        $\equiv \bot$ \\
        
        \InduktionsVoraussetzung
        Seien $\psi_1$, $\psi_2 \in$ AL, und seien $\psi_1'$, $\psi_2' \in$ AL$_1$ äquivalente Formeln für diese, sodass gilt:\\
        \begin{center}
            $\psi_1 \equiv \psi_1'$ und  $\psi_2 \equiv \psi_2'$
        \end{center}
        
        \newpage
        
        \InduktionsSchritt
        %%% Oder %%%
        Für $\varphi := \psi_1 \lor \psi_2$, $\varphi \in$ AL\\
        definieren wir $\varphi' := \neg (\neg \psi_1' \land \neg \psi_2') \land \varphi_{\top}$, $\varphi' \in$ AL$_1$\\
        
        Daraus folgt:\\
        $\varphi' := \neg (\neg \psi_1' \land \neg \psi_2') \land \varphi_{\top}$
        $\equiv  \neg (\neg \psi_1' \land \neg \psi_2')$ 
        $\equiv  \psi_1' \lor \psi_2'$
        
        
        Durch (IV) folgt: $\varphi \equiv \varphi'$\\
        
        %%% Negation %%%
        Für $\varphi := \neg \psi_1$, $\varphi \in$ AL\\
        definieren wir $\varphi' := \neg (\psi_1' \land \psi_1' \land \psi_1') $, $\varphi' \in$ AL$_1$\\
        
        Daraus folgt:\\
        $\varphi' := \neg (\psi_1' \land \psi_1' \land \psi_1')$
        $\equiv \neg (\psi_1' \land \psi_1')$
        $\equiv \neg \psi_1'$

        Durch (IV) folgt: $\varphi \equiv \varphi'$\\
        
        Nach Korollar 2.41 im Vorlesungsskript (oder auch auf Seite 3/14 der Folien 3.5 der Woche 2) ist jede aussagenlogische Formel äquivalent zu einer Formel, die lediglich Variablen, $\top$, $\bot$, $\neg$ und $\lor$ beinhaltet.\\
        Wie bereits bewiesen, existiert so eine Formel in AL$_1$.\\
        
        Also existiert zu jeder Formel $\varphi \in$ AL eine äquivalente Formel $\varphi' \in$ AL$_1$. Somit ist AL$_1$ eine Normalform.
        
        \end{induktion}
    \end{enumerate}
\end{aufgabe}

\end{document}
