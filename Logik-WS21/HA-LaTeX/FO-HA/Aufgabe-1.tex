\documentclass[10pt,a4paper]{scrartcl}

%%% Eine Erklärung der meisten eingebundenen Pakete finden sie in der LaTeX-Vorlage für die freiwilligen Hausaufgaben.
\usepackage[utf8]{inputenc}
\usepackage{fullpage}
\usepackage[ngerman]{babel}
\usepackage{dsfont}
\usepackage{hyperref}
\hypersetup{
	colorlinks=true,
	linkcolor=black!20!blue,
	citecolor=black!20!blue,
	urlcolor=black!20!blue
}
\usepackage{booktabs}
\usepackage{stmaryrd}
\usepackage{amsmath}
\usepackage{amssymb}
\usepackage{ebproof}
\usepackage{tikz}
\usepackage{algorithm}
\usepackage{forloop}
\usepackage{MnSymbol}
\usepackage[noend]{algpseudocode}

\usetikzlibrary{calc,through,arrows,automata,decorations.pathmorphing,decorations.pathreplacing,shapes,backgrounds,fit,positioning,shapes.geometric,patterns}

\usepackage{xstring}
\usepackage{xspace}

\usepackage{enumitem}
\setlist[enumerate]{label=(\roman*)}


%%% Zusätzliche Befehle %%%
\newcommand{\N}{\ensuremath{\mathds{N}}}
\newcommand{\Z}{\ensuremath{\mathds{Z}}}
\newcommand{\Q}{\ensuremath{\mathds{Q}}}
\newcommand{\R}{\ensuremath{\mathds{R}}}
\newcommand{\AL}{\ensuremath{\mathrm{AL}}\xspace}
\newcommand{\AVAR}{\textsc{AVar}}
\newcommand{\VAR}{\textsc{Var}}
\newcommand{\EF}{Ehrenfeucht-Fra\"{i}ss\'{e}\xspace}
\newcommand*{\FO}[1]{\ifx\relax #1\relax
	\ensuremath{\mathrm{FO}}\xspace
	\else
	\ensuremath{\mathrm{FO}[#1]}\xspace
	\fi}
\newcommand*{\Eval}[2]{\ensuremath{\left\llbracket #1 \right\rrbracket}^{#2}}
\newcommand{\sequent}{\Rightarrow}
\newcommand*{\axiom}[2]{\overline{#1 \overset{\phantom{.}}{\sequent} #2}}
% Makros für Induktionsbeweise
\newenvironment{induktion}
{
	\newif\ifIA
	\newif\ifIV
	\newif\ifIS
	\IAfalse
	\IVfalse
	\ISfalse
}
{
	\ifIA
	\else
	\errmessage{Der Induktionsanfang fehlt}
	\fi
	\ifIV
	\else
	\errmessage{Die Induktionsvoraussetzung fehlt}
	\fi
	\ifIS
	\else
	\errmessage{Der Induktionsschritt fehlt}
	\fi
}
\newcommand{\InduktionsAnfang}[1][]{\IAtrue \noindent\textbf{IA:} #1\\}
\newcommand{\InduktionsVoraussetzung}[1][]{\IVtrue \noindent\textbf{IV:} #1\\}
\newcommand{\InduktionsSchritt}[1][]{\IStrue \noindent\textbf{IS:}
	\ifx\relax #1\relax
	\else
	zu zeigen: #1
	\fi\\}
% Kiste für die Punkte der Teilaufgaben
	\newcounter{Iterator}
	\newcounter{Aux}
\newcommand{\PunkteKiste}[1]{%
	\newcommand{\CellHeight}{1}
	\newcommand{\CellWidth}{1.15}
	\begin{tikzpicture}
		\setcounter{Aux}{#1}
		\stepcounter{Aux}
		\forloop{Iterator}{1}{\value{Iterator} < \value{Aux}}{
			\node (v) at (\CellWidth * \value{Iterator}, \CellHeight+0.3) {({\roman{Iterator}})};%
			\draw[line width=0.5pt] (\CellWidth * \value{Iterator}+0.5, \CellHeight) -- (\CellWidth * \value{Iterator}+0.5,0);
			}
		\node (f) at (\CellWidth * \value{Aux}, \CellHeight+0.3) {Form};
		\draw[line width=0.5pt] (\CellWidth * \value{Aux} + 0.5, 0) -- (\CellWidth * \value{Aux} + 0.5, \CellHeight);
		\stepcounter{Aux}
		\node (s) at (\CellWidth * \value{Aux}, \CellHeight+0.3) {$\sum$};
		\draw[line width=0.5pt] (0.5 * \CellWidth, 0) -- (0.5 * \CellWidth, \CellHeight);
		\draw[line width=0.5pt] (\CellWidth * \value{Aux} + 0.5, 0) -- (\CellWidth * \value{Aux} + 0.5, \CellHeight);
		\draw[line width=0.5pt] (0.5 * \CellWidth, \CellHeight) -- (\CellWidth*\value{Aux}+0.5,\CellHeight);
		\draw[line width=0.5pt] (0.5 * \CellWidth, 0) -- (\CellWidth*\value{Aux}+0.5,0);
	\end{tikzpicture}
	}
% Aufgabe Umgebung
\newcounter{aufgabeNum}
\newenvironment{aufgabe}[1]
{

	\noindent\textbf{\underline{Aufgabe \refstepcounter{aufgabeNum}\arabic{aufgabeNum}}}\hfill\PunkteKiste{#1}

}
{}


%%% ANFANG DECKBLATT %%%

\begin{document}
\noindent
\noindent \large WiSe 21/22 \hfill
Logik

\begin{center}
\textbf{Hausarbeit 2 Aufgabe 1}\\
\vspace*{0.5cm}
\textbf{Gruppe:} 402355, 392210, 413316, 457146  % Tragen Sie hier die Matrikelnummern der Mitglieder ihrer Gruppe ein.
\end{center}

\newpage

%%% ENDE DECKBLATT %%%

\section*{Lösungen}
\begin{aufgabe}{10}
    \begin{enumerate}
    	\item
    	Gaifmangraphen von $\mathcal{B}$ :
    	\\
    	\begin{tikzpicture}[scale=1]
		\node [circle, draw](A0) at (0,0) {$2$};
		\node [circle, draw](A1) at (2,-1) {$6$};
		\node [circle, draw](A2) at (4,0) {$3$};
		\node [circle, draw](A3) at (2,-2) {$1$};
		\node [circle, draw](A4) at (0,-2) {$4$};
		\node [circle, draw](A5) at (4,-2) {$5$};
		\node [circle, draw](A6) at (2,-3) {$7$};
        
        \path
        (A0) edge (A1)
        (A0) edge (A2)
        (A1) edge (A2)
        (A0) edge (A3)
        (A1) edge (A3)
        (A2) edge (A3)
        (A4) edge (A3)
        (A5) edge (A3)
        (A6) edge (A3)
        (A0) edge (A4)
        ;
	\end{tikzpicture}
    	\item
    	
    	$ \mathcal{C} := ( [4], \: \mathcal{R}^\mathcal{C} := \{(a,b,c) \in [4]^3 \: | \: a.b = c\} ) $
        \\

        $ \mathcal{D} := ( [4], \: \mathcal{R}^\mathcal{D} := \{(a,b,c) \in [4]^3 \: | \: a/b = c\} ) $\\
        
        $\mathcal{R}^\mathcal{C} = \{(1,2,2)(2,1,2)(1,3,3)(3,1,3)(1,4,4)(4,1,4)(2,2,4)\} \\ \mathcal{R}^\mathcal{D} = \{(2,1,2)(3,1,3)(4,1,4)(4,2,4)\}$
        
        $\mathcal{R}^\mathcal{C} \not \cong \matDcal{R}^\mathcal{D}$
        
        $\mathcal{G}(\mathcal{C}) \cong \mathcal{G}(\mathcal{D})$
        
        \begin{tikzpicture}[node distance={15mm}, thick, main/.style = {draw, circle}]
        
            \node[main] (1) {$1$};
            \node[main] (2) [below left of=1] {$2$};
            \node[main] (3) [right  of=1] {$3$}; 
            \node[main] (4) [below right  of=1] {$4$};
        
            \draw (1) -- (2);
            \draw (1) -- (3);
            \draw (1) -- (4);
            \draw (2) -- (4);
        
        
        \end{tikzpicture}
        

    	\item
    	Die Struktur $I_{\psi_1}(G_1)$ sieht wie folgt aus:\\
    	\\
    	\begin{tikzpicture}[scale=1]
		\node [circle, draw](A0) at (0,0) {$b$};
		\node [circle, draw](A1) at (2,-0.75) {$g$};
		\node [circle, draw](A2) at (4,0) {$d$};
		\node [circle, draw](A3) at (2,-2) {$e$};

		\node [circle, draw](B0) at (6,0) {$a$};
		\node [circle, draw](B1) at (8,-0.75) {$f$};
		\node [circle, draw](B2) at (10,0) {$c$};
		\node [circle, draw](B3) at (8,-2) {$h$};
        
        \path
        (A0) edge (A1)
        (A0) edge (A2)
        (A1) edge (A2)
        (A0) edge (A3)
        (A1) edge (A3)
        (A2) edge (A3)
        
        (B0) edge (B1)
        (B0) edge (B2)
        (B1) edge (B2)
        (B0) edge (B3)
        (B1) edge (B3)
        (B2) edge (B3)
        ;
	\end{tikzpicture}
    	\item
    	 $\psi_2 := \exists z \exists w ((E(x,z) \wedge E(y,z)) \wedge (E(x,w) \wedge E(y,w)) \wedge (z \not = w \wedge \neg E(z,w)) )$
\\
        \item
        $\psi_3(u,v) := \exists z(R(u,v,z) \vee R(u,z,v) \vee R(z,u,v))$ \\
        Da die Formel sich auf $\tau$-Signaturen bezieht, können wir nur das Dreistellige Relationssymbol R nutzen. Wir suchen durch jede mögliche Kombination aus R-Relationen die u und v beinhalten. Wenn es eine davon gibt, geben wir das Tupel (u,v) aus.
    	\item
    	$\varphi_1(a,b) := E(a,b) \wedge E(b,a)$
    	\item
    	Die Aussage ist Falsch.\\
    	Angenommen G $\equiv_m$ H mit m $>$ 0 gilt, dann muss dies auch gelten wenn G $\nequiv_0$ H.
    	Sind G und H nicht 0-äquivalent und berechnen nun I$_\varphi$(G) und I$_\varphi$(H) für alle $\varphi \in $ FO[$\delta$] mit qr($\varphi$) $=$ 0. Dann sind I$_\varphi$(G) und I$_\varphi$(H) nicht äquivalent zueinander. Laut der Definition von m-äquivalenz können sie somit auch nicht für alle m $>$ 0 äquivalent zueinander sein.
    	\item
    	Die Aussage Stimmt.\\
    	Seien $I_{\varphi}(G) \equiv_m I_{\varphi}(H)$ mit m $>$ 0 und qr($\varphi) =$ 0. Dann gilt dadurch, dass $\varphi$ nur mit einem zweistelligen Relationssymbol definiert ist, dass $I_{\varphi}(G) \equiv_m I_{\varphi}(H)$ auch auf m $=$ 0 äquivalent sein müssen. Daher sind $I_{\varphi}(G)$ und $ I_{\varphi}(H)$ elementär äquivalent.\\
    	Laut "Tut 10 Lösung 3iii" gilt, dass wenn $I_{\varphi}(G) \equiv I_{\varphi}(H)$ gilt, dann gilt auch $I_{\varphi}(G) \cong I_{\varphi}(H)$ auf allen $\varphi_a \in$ FO[$\delta$].\\
    	Es gibt also eine Formel $\varphi_2 \in$ FO[$\delta$] für die gilt, $I_{\varphi_2}(I_{\varphi}(G)) =$ G, dann gilt auch  $I_{\varphi_2}(I_{\varphi}(H)) =$ H. Somit gilt die Aussage wenn $I_{\varphi}(G) \equiv_m I_{\varphi}(H)$, dann auch $G \equiv_m H$.\\
     	\item
     	
IA: Angenommen T ist ein Baum mit $|V(T)|$ $=$ n $=$ 2, dann ist $|E(T)|$ $=$ 1. Dann ist T' ein Graph mit $|V(T')|$ $=$ n $=$ 2 und $|E(T')|$ $=$ 1. Daraus folgt es gibt ein Hamiltonpfad von u nach v der Länge 2.\\

IV: Für ein festen $n \in \mathbb{N}$ mit $n \geq 2$ hat ein aus der Baum T konstruierten Graphen T' ein Hamiltonpfad für alle $\{u,v\} \in V(T')$ \\

IS: Laut $\chi(x,y)$ gilt: T ist ein Teilgraph von T'. Für alle $\{u,v\} \in E(T)$ existiert auch $\{u,v\} \in E(T')$. Für alle $\{u,z\},\{z,v\} \in E(T)$ existiert auch $\{u,v\} \in E(T')$, was bedeuted, dass T' einen vollständigen Teilgraph ($K_3$) der Länge 3 hat.
Für alle $\{u,z\},\{z,w\},\{w,v\} \in E(T)$ existiert auch $\{u,v\},\{u,w\},\{z,v\} \in E(T')$, was bedeuted, dass T' einen vollständigen Teilgraph ($K_4$) der Länge 4 hat.

Daraus folgt, dass T' enthält eine Menge aus vollstendigen $K_3$ und $K_4$ Teilgraphen, und jeder Knoten $x \in V(T')$ teil mindestens einen solchen vollständigen Teilgraph ist. (Für $n \geq 3$ sind alle $x \in V(T')$ immer Teil mindestens einen vollständigen $K_4$ Teilgraph).

Da jeder vollständigen Graph einen Hamiltonpfad hat, folgt, dass für alle $\{u,v\} \in E(T)$ einen Hamiltonpfad in T' existiert.

\item
    	Aus Aufgabe ix) können wir ableiten, dass es einen Hamiltonpfad in $H^'$ gibt von u nach v für jedem Zusammenhängenden Graphen $H$, da durch $\chi$ alle 3er und 4er Knotenpaare einen Kreis bilden. somit gibt es auch ein 3er Kreis mit (u,v,x) wobei x $\in$ V(H). Durch den Hamiltonpfad von u nach v und der Kante E(u,v) existiert somit auch ein Hamiltonkreis in $H^'$
        
    \end{enumerate}
\end{aufgabe}



\end{document}
