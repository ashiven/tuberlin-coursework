\documentclass[10pt,a4paper]{scrartcl}

%%% Eine Erklärung der meisten eingebundenen Pakete finden sie in der LaTeX-Vorlage für die freiwilligen Hausaufgaben.
\usepackage[utf8]{inputenc}
\usepackage{fullpage}
\usepackage[ngerman]{babel}
\usepackage{dsfont}
\usepackage{hyperref}
\hypersetup{
	colorlinks=true,
	linkcolor=black!20!blue,
	citecolor=black!20!blue,
	urlcolor=black!20!blue
}
\usepackage{booktabs}
\usepackage{stmaryrd}
\usepackage{amsmath}
\usepackage{amssymb}
\usepackage{ebproof}
\usepackage{tikz}
\usepackage{algorithm}
\usepackage{forloop}
\usepackage{MnSymbol}
\usepackage[noend]{algpseudocode}

\usetikzlibrary{calc,through,arrows,automata,decorations.pathmorphing,decorations.pathreplacing,shapes,backgrounds,fit,positioning,shapes.geometric,patterns}

\usepackage{xstring}
\usepackage{xspace}

\usepackage{enumitem}
\setlist[enumerate]{label=(\roman*)}


%%% Zusätzliche Befehle %%%
\newcommand{\N}{\ensuremath{\mathds{N}}}
\newcommand{\Z}{\ensuremath{\mathds{Z}}}
\newcommand{\Q}{\ensuremath{\mathds{Q}}}
\newcommand{\R}{\ensuremath{\mathds{R}}}
\newcommand{\AL}{\ensuremath{\mathrm{AL}}\xspace}
\newcommand{\AVAR}{\textsc{AVar}}
\newcommand{\VAR}{\textsc{Var}}
\newcommand{\EF}{Ehrenfeucht-Fra\"{i}ss\'{e}\xspace}
\newcommand*{\FO}[1]{\ifx\relax #1\relax
	\ensuremath{\mathrm{FO}}\xspace
	\else
	\ensuremath{\mathrm{FO}[#1]}\xspace
	\fi}
\newcommand*{\Eval}[2]{\ensuremath{\left\llbracket #1 \right\rrbracket}^{#2}}
\newcommand{\sequent}{\Rightarrow}
\newcommand*{\axiom}[2]{\overline{#1 \overset{\phantom{.}}{\sequent} #2}}
% Makros für Induktionsbeweise
\newenvironment{induktion}
{
	\newif\ifIA
	\newif\ifIV
	\newif\ifIS
	\IAfalse
	\IVfalse
	\ISfalse
}
{
	\ifIA
	\else
	\errmessage{Der Induktionsanfang fehlt}
	\fi
	\ifIV
	\else
	\errmessage{Die Induktionsvoraussetzung fehlt}
	\fi
	\ifIS
	\else
	\errmessage{Der Induktionsschritt fehlt}
	\fi
}
\newcommand{\InduktionsAnfang}[1][]{\IAtrue \noindent\textbf{IA:} #1\\}
\newcommand{\InduktionsVoraussetzung}[1][]{\IVtrue \noindent\textbf{IV:} #1\\}
\newcommand{\InduktionsSchritt}[1][]{\IStrue \noindent\textbf{IS:}
	\ifx\relax #1\relax
	\else
	zu zeigen: #1
	\fi\\}
% Kiste für die Punkte der Teilaufgaben
	\newcounter{Iterator}
	\newcounter{Aux}
\newcommand{\PunkteKiste}[1]{%
	\newcommand{\CellHeight}{1}
	\newcommand{\CellWidth}{1.15}
	\begin{tikzpicture}
		\setcounter{Aux}{#1}
		\stepcounter{Aux}
		\forloop{Iterator}{1}{\value{Iterator} < \value{Aux}}{
			\node (v) at (\CellWidth * \value{Iterator}, \CellHeight+0.3) {({\roman{Iterator}})};%
			\draw[line width=0.5pt] (\CellWidth * \value{Iterator}+0.5, \CellHeight) -- (\CellWidth * \value{Iterator}+0.5,0);
			}
		\node (f) at (\CellWidth * \value{Aux}, \CellHeight+0.3) {Form};
		\draw[line width=0.5pt] (\CellWidth * \value{Aux} + 0.5, 0) -- (\CellWidth * \value{Aux} + 0.5, \CellHeight);
		\stepcounter{Aux}
		\node (s) at (\CellWidth * \value{Aux}, \CellHeight+0.3) {$\sum$};
		\draw[line width=0.5pt] (0.5 * \CellWidth, 0) -- (0.5 * \CellWidth, \CellHeight);
		\draw[line width=0.5pt] (\CellWidth * \value{Aux} + 0.5, 0) -- (\CellWidth * \value{Aux} + 0.5, \CellHeight);
		\draw[line width=0.5pt] (0.5 * \CellWidth, \CellHeight) -- (\CellWidth*\value{Aux}+0.5,\CellHeight);
		\draw[line width=0.5pt] (0.5 * \CellWidth, 0) -- (\CellWidth*\value{Aux}+0.5,0);
	\end{tikzpicture}
	}
% Aufgabe Umgebung
\newcounter{aufgabeNum}
\newenvironment{aufgabe}[1]
{

	\noindent\textbf{\underline{Aufgabe \refstepcounter{aufgabeNum}\arabic{aufgabeNum}}}\hfill\PunkteKiste{#1}

}
{}


%%% ANFANG DECKBLATT %%%

\begin{document}
\noindent
\noindent \large WiSe 21/22 \hfill
Logik

\begin{center}
\textbf{Hausarbeit 2 Aufgabe 2}\\
\vspace*{0.5cm}
\textbf{Gruppe:} 402355, 392210, 413316, 457146  % Tragen Sie hier die Matrikelnummern der Mitglieder ihrer Gruppe ein.
\end{center}

\newpage

%%% ENDE DECKBLATT %%%

\section*{Lösungen}
\setcounter{aufgabeNum}{1}
\begin{aufgabe}{2}
    \begin{enumerate}
    	\item
    	$\varphi_1$ fordert, dass es für jedes Element $a$ ein Element $b$ gibt, sodass $E(b,a)$ und $\neg E(a,a)$ gilt.\\
    	$\varphi_2$ fordert, dass jedes Element $a$, das in Relation $E(a,b)$ steht, nur mit genau einem weiteren Element $b$ in Relation $E(a,b)$ steht.\\
    	$\varphi_3$ fordert, dass die Relation $E(a,b)$ symmetrisch ist.\\
    	$\varphi_4$ fordert, dass $F(a)$ gilt, genau dann wenn ein Element $a$ mit keinem Element $b$ in Relation $E(a,b)$ steht und dass ein Element $a$ mit keinem Element $b$ in Relation $E(a,b)$ steht, genau dann wenn $F(a)$ gilt.\\
    	$\varphi_5$ fordert, dass gilt $E(a,b)$ und $E(c,d)$ mit $c \neq a$ und $c \neq b$, sodass für alle weiteren Elemente $e$, die nicht in $\{a, b, c, d\}$ enthalten sind, $F(e)$ gilt.\\
    	
    	endliches Modell: Wir interpretieren $E$ als eine symmetrische Kantenrelation, bei der die Kante in beide Richtungen gerichtet ist. Sei $\mathcal{A}$ eine $\sigma$-Struktur mit Universum 
    	$A := \{1, 2, 3, 4\}$\\ 
    	mit $E^\mathcal{A} = \{(1,2), (3,4)\}$ und $F^\mathcal{A} = \{\}$\\
    	
    	$\varphi_1$ wird erfüllt, da für jedes Element $a$ in unserem Universum gilt, es steht mit einem anderen Element $b$ in unserem Universum in einer nicht reflexiven Relation $E(a,b)$. Da wir $E$ als eine Kantenrelation interpretieren, bei der die Kante in beide Richtungen zeigt, gilt auch $E(b,a)$. Somit steht jedes Element $a$ mit einem anderen Element $b$ in Relation $E(a,b)$.\\
    	$\varphi_2$ wird erfüllt, da jedes Element $a$ in unserem Universum nur mit genau einem weiteren Element $b$ in der symmetrischen Relation $E(a,b)$ steht.\\
    	$\varphi_3$ wird erfüllt, da wir $E$ als symmetrische Kantenrelation interpretieren. Somit gilt für alle Elemente $a, b$ in unserem Universum, die in Relation $E(a,b)$ stehen, auch $E(b,a)$.\\
    	$\varphi_4$ wird erfüllt, da $F^\mathcal{A}$ leer ist und somit wird $F(a)$ zu 0 ausgewertet für alle Elemente $a$ in unserem Universum gilt und da jedes $a$ in unserem Universum in einer Relation $E(a,b)$ mit einem weiteren Element $b$ steht. Somit werden beide Seiten der Biimplikation zu 0 ausgewertet und $\varphi_4$ wird erfüllt.\\
    	$\varphi_5$ wird erfüllt, da wir vier untereinander unterschiedliche Elemente haben, die in Relation $E(1,2)$ und $E(3,4)$ stehen. Somit gibt es ein $a$ und ein $b$, sodass $E(a,b)$ und es gibt ein $c$ und ein $d$, sodass $E(c,d)$ mit $c \neq a$ und $c \neq b$. Da $F^\mathcal{A}$ leer ist, wird die rechte Seite der Implikation zu 0 ausgewertet und da wir kein fünftes Element $e$ im Universum haben, das nicht in der Menge $\{1, 2, 3, 4\}$ enthalten ist, wird die große Verorderung immer zu 1 ausgewertet und dann durch die Negation zu 0. Somit werden beide seiten der Implikation zu 0 ausgewertet und die Implikation dann zu 1 und $\varphi_5$ wird erfüllt.\\ 
    	Die endliche $\sigma$-Struktur $\mathcal{A}$ erfüllt $\Phi$.\newpage
    	
    	Die Formelmenge $\Phi$ wird von keiner $\sigma$-Struktur mit weniger oder mehr als 4 Elementen erfüllt.\\
    	$\varphi_5$ fordert, dass gilt $E(a,b)$ und $E(c,d)$ mit $c \neq a$ und $c \neq b$.\\
    	Demnach haben wir 4 Elemente $\{a, b, c, d\}$, die in Relation $E(a,b)$ und $E(c,d)$ stehen.\\
        
        Aus den Interpretationen von $\varphi_1$, $\varphi_2$ und $\varphi_3$ folgt, dass jedes Element $a$, in genau einer symmetrischen, nicht reflexiven Relation $E(a,b)$ mit genau einem weiteren Element $b$ steht. Demnach müssen unsere 4 Elemente $a, b, c, d$ in $\varphi_5$ untereinander verschieden sein, da sie in den Relationen $E(a,b)$ und $E(c,d)$ stehen. Demnach benötigt unsere $\sigma$-Struktur mindestens 4 Elemente.\\
    	Da jedes Element $a$ in einer symmetrischen, nicht reflexiven Relation $E(a,b)$ mit genau einem weiteren Element $b$ steht, folgt daraus, dass die linke Seite der Biimplikation in $\varphi_4$ immer zu 0 ausgewertet wird. Daraus folgt, dass $F(a)$  für jedes Element $a$ zu 0 ausgewertet werden muss, damit $\varphi_4$ erfüllt wird.\\
    	Für $\varphi_5$ bedeutet das, dass die rechte Seite der Implikation immer zu 0 ausgewertet wird. Damit $\varphi_5$ erfüllt wird, muss die linke Seite der Implikation ebenfalls zu 0 ausgewertet werden. Dazu darf die große Veroderung nicht zu 0 ausgewertet werden, da sie sonst durch die Negation zu 1 ausgewertet wird. Das bedeutet, dass es kein Element $e$ geben darf, dass nicht in $\{a, b, c, d\}$ enthalten ist. Daraus folgt, dass wir nicht mehr als 4 Elemente im Universum haben dürfen.\\
    	
    	Daraus folgt, $\Phi$ kann nur mit einer $\sigma$-Struktur mit genau 4 Elementen erfüllt werden. Demnach besitzt $\Phi$ kein unendliches Modell.
    	
    	\item
    	Angenommen es existiert eine Formel $\varphi \in$ FO$[\sigma]$ fuer die gilt: $\varphi(\mathcal{Q}) = \N$.\\
    	Die Formel $\varphi$ waere wie folgt aufgebaut:\\
    	
    	Terme $\tau$:\\
    	\bullet $x,y \in$ $\Q$\\
    	\bullet $\odot_{\mathcal{Q}}(a,b)$, $a,b \in \tau$\\
    	
    	Formeln FO$[\sigma]$:\\
    	\bullet $t $= $ t'$, $t,t' \in \tau$\\
    	\bullet $\lnot \varphi$, $\varphi \in$ FO$[\sigma]$\\
    	\bullet $\varphi * \psi$, $\varphi, \psi \in$ FO$[\sigma]$, $* \in \{\land, \lor, \rightarrow, \leftrightarrow\}$\\
    	\bullet $\exists x \varphi(x), \forall y \varphi(y)$, $x,y \in$ $\Q$, $\varphi \in$ FO$[\sigma]$\\
    	
    	Wir betrachten im folgenden eine Fallunterscheidung fuer die atomaren Formeln $t = t'$.\\
    	
    	Fall 1: $t = x$, $t' = y$\\
    	$t = t'$ ist erfuellt wenn $x = y$ gilt und erfuellbar fuer $x,y \in \Q$\\
    	
    	Fall 2: $t = x$, $t' = \odot_{\mathcal{Q}}(y,z)$\\
    	$t = t'$ ist erfuellt wenn $x = y + z$ gilt und erfuellbar fuer $x,y,z \in \Q$\\
    	
    	Fall 3: $t = \odot_{\mathcal{Q}}(a,b)$, $t' = \odot_{\mathcal{Q}}(c,d)$\\
    	$t = t'$ ist erfuellt wenn $a + b = c + d$ gilt und erfuellbar fuer $a,b,c,d \in \Q$\\
        
        Jede Formel $\varphi \in$ FO$[\sigma]$ wird aufbauend auf den atomaren Formeln konstruiert und es gelten somit die selben Erfuellbarkeiten.
        Dies ist ein Widerspruch zu unserer Annahme, da wir somit keine Formel konstruieren koennen welche fuer alle $x \in \N$ jedoch fuer kein $y \in \Q \setminus \N$ erfuellbar ist.
 
        
    \end{enumerate}
\end{aufgabe}

\end{document}
